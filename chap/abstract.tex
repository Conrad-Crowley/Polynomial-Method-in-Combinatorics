\begin{abstract}
    Polynomial methods are an emerging set of techniques in Combinatorics which use elementary facts about polynomials to control the size of collections of objects with a certain structure. We present an exposition of results proven using polynomial methods,
    beginning with Dvir's resolution of the Finite Field Kakeya Problem. We then explore the first application of a polynomial method in Combinatorics, Alon's proof of the Cauchy-Davenport theorem via the Combinatorial Nullstellensatz. These methods have proven
    most fruitful in the field of Incidence Geometry, and we explore the application of these methods to the Joints Problem, the Szemerédi-Trotter Theorem, and the problem of controlling circle tangencies in the plane.  
    
    In the latter problem we present a new proof for controlling the number of tangencies in a (non-degenerate) collection of  circles by $O(N^{3/2})$, utilising Guth's recent result of polynomial partitioning for varieties.
\end{abstract}