\chapter{Cauchy-Davenport Theorem \label{chap:alon}}
In this chapter we shall discuss applications to the exciting field of additive combinatorics, where in 1990 Alon provided perhaps
the first example of a polynomial method in action (see \cite{alon1999combinatorial}). This chapter is distinct from the other chapters in this manuscript as it deals with applications outside a geometric setting. It showcases a different polynomial method to the previous chapter, employing instead a combinatorial version of Hilbert's Nullstellensatz.
In \cite{michalek2010}, Michalek produced a short, elementary, and direct proof of the combinatorial Nullstellensatz which we present here.
\section{Combinatorial Nullstellensatz}
The German term \textit{Nullstellensatz} means ``theorem about zeros'', so it should come as no surprise that the theorem is 
precisely a statement about the general size and shape of sets of zeros of a polynomial depending on its highest degree terms. 

The main idea behind the Nullstellensatz comes from the fact that a polynomial in one variable of degree $D$ cannot have more than $D$ roots. 
As a consequence of this if we have a set of $D+1$ elements they cannot all be roots of the same degree $D$ polynomial. In the univariate case this is somewhat elementary, but it has a very non-trivial extension to grids of points in arbitrary dimensions. 

\begin{theorem}[Combinatorial Nullstellensatz]
    Let $\KK$ be a (not necessarily finite) field, and let $P(X_1,\dots, X_n) \in \KK[X_1,\dots, X_n]$ be a polynomial in $n$ variables with coefficients in $\KK$. Suppose the coefficient of $X_1^{k_1}X_2^{k_2}\dots X_n^{k_n}$ is non-zero and further suppose $\deg P = \sum_{i=1}^n k_i$, where each $k_i$ is a non-negative integer.

    Then for any subsets $A_1,\dots A_n$ of $\KK$ satisfying $|A_i| > k_i$ for each $1\leq i\leq n$ there exist $a_1 \in A_1, \dots, a_n \in A_n$ such that $P(a_1,\dots, a_n) \neq 0$.
    \label{thm:comb-nullstellensatz}
\end{theorem}
\begin{proof}
    We proceed by induction on $\deg P = D$. When $D=1$, $P$ is simply a linear combination of $n$ variables say $P(X_1,\dots X_m) = c_1 X_1 + \dots c_n X_n$. Without loss of generality assume $X_1$ has a non-zero coefficient and consider the sets $A_i = \{a_{i}, b_{i}\}$. Suppose $P$ at the point $(a_{1}, a_{2}, \dots, a_{n})$ is zero. We can then determine 
    \[c_{1} =- \frac{c_2 a_{2} + c_3 a_{3}+\dots+ c_n a_{n}}{a_{1}}.\] 
    Now evaluating at $(b_1, a_{2}, \dots, a_{n})$ we see that this is zero only when $a_{1} = b_{1}$, so our theorem holds. 

    
    Now let us assume the theorem holds for $\deg P = D -1$, and prove it for $\deg P = D$.
    Suppose that $P$ satisfies the assumptions of the theorem but $P(x) = 0$ for every $x \in A_1 \times \dots \times A_n$.
    Without loss of generality $k_1 > 0.$ Fixing $a\in A_1$ we can write
    \begin{align}
        P = (x_1-a)Q +R \label{eq:alon-fact}
    \end{align}
    by the usual long division of polynomials. The degree of $R$ in $x_1$ must be strictly less than $\deg(X_1-a)$, so $R$ is independent of $X_1$. Thus it follows that $Q$ must have a monomial with non-zero coefficient of the form $X_1^{k_{1} -1} X_2^{k_2} \dots X_n^{k_n}$ and 
    $\deg (Q) = D-1$.

    Take any $x \in \{a\} \times A_2 \times \dots \times A_n$ and evaluate $(\ref{eq:alon-fact})$. Since $P(x) = 0$ it follows that $R(x) = 0$. Hence $R$ vanishes identically on the slice $\{a\} \times A_2 \times \dots \times A_n$. Since $R$ is independent of $X_1$ it must also vanish identically on $A_1 \times A_2 \times \dots \times A_n$.
    Now take any $x \in A_1 \backslash \{a\} \times A_2 \times \dots \times A_n$ and evaluate $(\ref{eq:alon-fact})$. Since the $(X_1 - a)$ term is non-zero, $Q(x) =0$. So $Q$ vanishes on all $x \in A_1 \backslash \{a\} \times A_2 \times \dots \times A_n$, which contradicts the inductive hypothesis.
\end{proof}
\section{Proof of Cauchy-Davenport Theorem}
To showcase the usefulness of the Combinatorial Nullstellensatz, we shall prove a classical result in Additive Combinatorics. 
\begin{theorem} [Cauchy-Davenport Theorem] \label{thm:C-D}
    Let $A,B$ be non-empty subsets of $\ZZ_p$ for some $p$ prime. Define their sumset $A+B$ as follows:
    \[
    A + B = 
    \left\{ x \in \ZZ_p\ | \ x = a+b \text{  for some } a\in A, \ b \in B \right\}.
    \]
    Then we have:
    \[
    |A+B| \geq  \min \left\{p, |A| + |B| -1 \right\}.
    \]
\end{theorem}

\begin{proof}
    Let us tackle the two cases separately. First, assume that $\min \left\{p, |A| + |B| -1 \right\} = p$.
    Then if $|A| + |B| > p$, $A$ and $B$ must intersect by the pigeonhole principle.
    For any $g \in \ZZ_p$ denote the set  $\left\{ g - x  \ | \ x \in B \  \right\} \subset \ZZ_p$ as $g-B$. Since $|g-B| = |B|$, we have that 
    $g-B$ and $A$ must intersect as well. Thus there exists some $a \in A$, $b \in B$ such that: 
    \begin{align*}
        g& -b = a \\
        g& = a+b.
    \end{align*}
    Our choice of $g$ was arbitrary, so it follows that $A+B = \ZZ_p$ and hence $|A+B| = p$.

    Now assume that $\min \left\{p, |A| + |B| -1 \right\} = |A| + |B| -1$. Then if the theorem is false we have $|A+B| \leq |A| + |B| -2$, so there exists some $C \subset \ZZ_p$ such that 
    $A+B \subset C$ and $|C| =  |A| + |B| -2$. Now let us define a polynomial $f(X,Y) \in \ZZ_p [X,Y]$ as
    \[
        f(X,Y) = \prod_{c \in C} (X+Y -c).
    \]
    Since $A+B \subset C$, $f(a,b) =0$ for all $(a,b) \in A\times B$. Further, the degree of $f$ is $\deg f = |C| = |A| + |B| -2$. 
    We can now appeal to the \hyperref[thm:comb-nullstellensatz]{Combinatorial Nullstellensatz} to yield a contradiction. Let $k_1 = |A| -1$, and $k_2 = |B|-1$. 
    Now $\deg f = k_1 + k_2$, and the coefficient of $x^{k_1}y^{k_2}$ is
    ${|A|+|B| - 2} \choose {|A|- 1}$, which is non-zero in $\ZZ_p$ as the binomial coefficient is not divisible by $p$ if all its factors are less than $p$. Applying Theorem \ref{thm:comb-nullstellensatz} we see that there 
    must exist some $(a,b) \in A \times B$ such that $f(a,b) \neq 0$, a contradiction.
\end{proof}

The Cauchy-Davenport Theorem (Theorem \ref{thm:C-D}) is indeed tight as illustrated by the following example.

\begin{example}
    Fix $b \in \ZZ_p$ and consider the subsets of $\ZZ_p$ given by
    \begin{align*}
        A &= \{t b \ | \ 1\leq t \leq m \},\\
        B &= \{t' b \ | \ 1\leq t' \leq n \}.
        \intertext{Notice that $|A|=m$ and $|B| = n$. Their sumset is given by}
        A+B &= \{(t+t') b \ | \ 1\leq t \leq m, \ 1\leq t' \leq m \}.
        \intertext{ The $(t+t')$ term takes on every value between 2 and $m+n$ at least once. Hence}
        |A+B| &= m + n -1 = |A| + |B| -1.
    \end{align*}
\end{example} 
