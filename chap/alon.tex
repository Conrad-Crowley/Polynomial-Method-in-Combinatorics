\chapter{The Polynomial Method in Additive Combinatorics}

\begin{theorem}[Combinatorial Nullstellensatz]
    Let $\KK$ be a (not necessarily finite) field, and let $P(x_1,\dots, x_n) \in \KK[X_1,\dots, X_n]$ be a polynomial in $n$ variables with coefficients in $\KK$. [TODO: Tautology?] Suppose $\deg P = \sum_{i=1}^n k_i$, where each $k_i$ is a non-negative
    integer, and further suppose the coefficient of $x_1^{k_1}x_2^{k_2}\dots x_n^{k_n}$ is non-zero.

    Then for any subsets $A_1,\dots A_n$ of $\KK$ satisfying $|A_i| > k_i$ for each $1\leq i\leq n$ there exist $a_1 \in A_1, \dots, a_n \in A_n$ such that $P(a_1,\dots, a_n) \neq 0$.
    
\end{theorem}
\begin{proof}
    We proceed by induction on $\deg P = D$. When $D=1$, $P$ is simply a linear combination of $n$ variables so the theorem holds.

    Now let us assume the theorem holds for $\deg P = D -1$, and prove for $\deg P = D$.
    Suppose that $P$ satisfies the assumptions of the theorem but $P(x) = 0$ for every $x \in A_1 \times \dots \times A_n$.
    Without loss of generality $k_1 > 0.$ Fixing $a\in A_1$ we can write
    \[
    P = (x_1-a)Q +R  \tag{$\dagger$}
    \]
    by the usual long division of polynomials. The degree of $R$ in $x_1$ must be strictly less than $\deg(x_1-a)$, so $R$ does not contain any
    $x_1$ terms. Thus it follows that $Q$ must have a monomial with non-zero coefficient of the form $x_1^{k_{1} -1} x_2^{k_2} \dots x_n^{k_n}$ and 
    $\deg (Q) = D-1$.

    Take any $x \in \{a\} \times A_2 \times \dots \times A_n$ and evaluate $(\dagger)$. Since $P(x) = 0$ it follows that $R(x) = 0$, but $R$ is independent of $x_1$ so $R$ must also vanish on $A_1 \backslash \{a\} \times A_2 \times \dots \times A_n$.
    Now take any $x \in A_1 \backslash \{a\} \times A_2 \times \dots \times A_n$ and evaluate $(\dagger)$. Since $(x_1 - a)$ is non-zero, $Q(x) =0$. So $Q$ vanishes on all $x \in A_1 \backslash \{a\} \times A_2 \times \dots \times A_n$, which contradicts the inductive hypothesis.
\end{proof}

\begin{theorem} [Cauchy-Davenport Theorem]
    Let $A,B$ be non-empty subsets of $\ZZ_p$ for some $p$ prime. Define their sumset as follows:
    \[
    A + B = 
    \left\{ x \in \ZZ_p\ | \ x = a+b \text{  for some } a\in A, \ b \in B, a\neq b. \right\}
    \]
\end{theorem}