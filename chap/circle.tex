\chapter{Counting Circle Tangencies\label{chap:circle}}

Here we shall discuss the problem of counting the number of tangencies in a suitably non-degenerate collection of circles. We say two circles are tangent if their intersection contains a single point.
The set of pairs of circles in a collection $\CE$ which are mutually tangent are called the tangencies and the collection of all tangencies is denoted $\tau(\CE)$. 

\begin{lemma}[Trivial Bound] \label{lem:trivial-circle-bound}
    Let $\CE$ be an arbitrary finite collection of circles. Then the number of tangencies $|\tau(\CE)|$ is bounded by
    \[
        |\tau(\CE)| \leq |\CE|^2.
    \] 
\end{lemma}
\begin{proof}
    For each pair of circles in $\CE$ there can be at most one tangency between them.
    Since there are $\leq |\CE|^2$ pairs, the bound follows.
\end{proof}
Stated for arbitrary collections of circles the problem is not particularly interesting as the above bound turns out to be asymptotically tight, as the following example shows.
\begin{example}
Denote a circle centred at $(x,y)$ with radius $r$ as $\gamma (x,y,r)$.
Consider the following collection of $N+1$ circles:
\begin{align*}
    C_0 &= \gamma\left(0,0,1\right) \\
    C_1 &= \gamma\left(\frac{1}{2},0,\frac{1}{2}\right) \\
    C_2 &= \gamma\left(\frac{3}{4},0, \frac{1}{4}\right) \\
    &\vdots \\
    C_N &= \gamma\left(1- \frac{1}{2^N}, 0, \frac{1}{2^N}\right)
\end{align*}
Each circle in our collection $\CE = \{ C_i \ | \ 0 \leq i \leq N\}$ is tangent to $N$ other circles at the point (1,0). 
Hence $|\tau(\CE)| \sim N^2 \sim |\CE|^2$. 
\begin{figure}[h]
    \centering 
    \includegraphics[width=0.8\textwidth]{images/circles_degen_case_ManimCE_v0.15.1.png}
    \caption{A collection of circles in $\RR^2$ with $N^2$ tangencies.}
    \end{figure}
\end{example}

In light of this example, in order to obtain a non-trivial example we shall look at collections of circles that satisfy a non-degeneracy condition. We consider collections of circles 
such that no three circles are mutually tangent at a common point. The best known examples for collections that satisfy this non-degeneracy condition leverage our asymptotically tight \hyperref[thm:S-T]{Szemerédi-Trotter Theorem} (Theorem \ref{thm:S-T}) examples:
\begin{example}
    Let $\ES$, $\LL$ be collections of $N$ points and $N$ lines respectively such that the \hyperref[thm:S-T]{Szemerédi-Trotter Theorem}'s bound is asymptotically tight. 
    In particular, they determine $\sim N^{4/3}$ point-line incidences. 
    Let $C_1$ denote the collection of unit circles centred at the points of $\ES$. 
    Let $C_2$ denote the collection of lines obtained by translating each line $\ell \in \LL$ one unit in the $\ell^{\perp}$ direction.
    If $(p,\ell) \in \ES \times \LL$ is a point-line incidence from our original collection, then the corresponding circle-line pair will be tangent.
    Performing an inversion transform about a point that does not lie in any of the circles or lines, our collection $\CE = C_1 \cup C_2$ becomes a 
    collection of $2N$ circles that determine at least $N^{4/3}$ tangencies. \label{ex:circle-lower-bound}
 \end{example}

We shall now present a bound from a recent paper of Ellenberg-Solymosi-Zahl which uses the polynomial method (See \cite{ellenberg2016new}).

\begin{theorem}[\cite{ellenberg2016new}]
    Given a finite collection $\CE$ of $N$ circles in the plane such that no three are mutually tangent at a common point, 
    then the number of tangencies $|\tau(\CE)|$ obeys   \label{thm:circle-tangencies}
    \[
        |\tau(\CE)| \lesssim N^{3/2}.
    \]
\end{theorem}

It is not obvious how to apply the polynomial method in the current formulation. 
We will perform a lifting  of our circles into algebraic curves in $\RR^3$, 
transforming tangencies between circles in the plane to incidences between their lifted curves in $\RR^3$. 
This transforms the problem from a tangency problem to an incidence problem which can therefore be attacked with the tools available to incidence geometry. As an additional benefit to this lifting, it reduces the required degree of polynomial needed to interpolate the incidences.
In general Lemma \ref{lem:paramcounting} tells us that if we are to interpolate a set of $M$ points in $\RR^d$ by a polynomial $P$, then the degree of $P$ is $O(M^{1/d})$, 
so increasing the dimension of the problem yields a lower degree of the polynomial required.

\section{Lifting of circles to $\RR^3$}

We shall now discuss this lifting in detail. Let $\gamma \subset \RR^2$ be a circle in the plane of radius $r_{\gamma}$ centred at $(x_\gamma, y_\gamma)$. 
We define the lifting transform of the circle as \[
    \beta(\gamma) = \left \{ (x,y,z) \in \RR^3 \ \big | \ (x-x_\gamma)^2 + (y-y_\gamma)^2 = r_\gamma^2, \ z = - \frac{x-x_\gamma}{y-y_\gamma} \right \}.
\]  
Clearly $\beta(\gamma)$ is an algebraic curve, so now we shall examine what happens to pairs of tangent circles under this transformation. Notice that $z$ is defined as the slope of the tangent line
at the point $(x,y) \in \gamma$.

\begin{lemma}
    Let $\beta$ be the transform defined as above. 
    Then two circles $\gamma, \gamma' \subset \RR^2$ are tangent if and only if $\beta(\gamma) \cap \beta(\gamma') \neq \emptyset$.    \label{lem:beta-lift}
\end{lemma}
\begin{proof}
If $\gamma$ and $\gamma'$ are tangent then there exists a point $(x,y) \in \gamma \cap \gamma'$ and
at a point of tangency we have that the slopes of the tangent lines coincide for $\gamma$ and $\gamma'$. Explicitly $-\frac{x-x_\gamma'}{y - y_\gamma'}= -\frac{x-x_\gamma}{y - y_\gamma} = z$.
Hence $(x,y,z) \in \beta(\gamma) \cap \beta(\gamma')$ and thus $ \beta(\gamma) \cap \beta(\gamma') \neq \emptyset$.

In the other direction, assume there exists some $(x,y,z) \in \beta(\gamma) \cap \beta(\gamma')$. 
Clearly $(x,y) \in \gamma \cap \gamma'$ and the slopes of the tangent lines at this point are equal,
hence we conclude $\gamma$ is tangent to $\gamma'$.
\end{proof}
This one-to-one correspondence between tangencies in $\RR^2$ and incidences in $\RR^3$ is the key idea behind the proof. 
It may be viewed as the graph of a Gaussian map in a certain coordinate system.

In the course of the proof we will need a Bézout-type result which bounds the number of intersections between a curve and a surface with no common components.
\begin{lemma}[Bézout Lemma in $\RR^3$]     \label{lem:Bezout}
   Let $Z(P)$ and $Z(Q_1,Q_2)$ be the zero sets of polynomials over $\RR[X,Y,Z]$ and suppose that $P,Q_1,Q_2$ have no pairwise common factors.
 Then we can bound the cardinality of the intersection of the zero sets by
    \[ |Z(P) \cap Z(Q_1,Q_2)| \lesssim \deg P \deg Q_1 \deg Q_2.\]
\end{lemma}
To preserve the flow of this chapter, the proof of this result will be given in Appendix \ref{appendix:Bezout}.

\section[Ellenberg-Solymosi-Zahl's Proof]{Ellenberg-Solymosi-Zahl's Proof of Theorem \ref{thm:circle-tangencies} \label{sect:Zahl-Proof}}
In this proof, we will need to ensure our tangencies
are sufficiently uniformly distributed among our circles. We can refine our collection such that this is the case by the following lemma.

\begin{lemma}[Uniform Refinement]
    Let $\CE$ be a collection of $N$ circles and suppose that $|\tau(\CE)| \gtrsim N^{\alpha}$. 
    Then we can refine our collection to a collection $\CE' \subset \CE$ such that every circle in $\CE'$ is tangent to $\gtrsim N^{\alpha -1}$ other circles in $\CE'$, $|\tau(\CE')| \gtrsim N^{\alpha}$, and $|\CE'| \gtrsim N^{\alpha/2}$. \label{lem:uniform-refine}
\end{lemma}

\begin{proof}
We proceed by a stopping-time argument.
Let $\tau(\CE)$ be the set of tangencies of the circles in $\CE$ and let $c_0$ be a constant such that $|\tau(\CE)| \geq c_0 N^{\alpha}$. Let $\CE_0 = \CE$ and let $c_1$ be a fixed constant to be chosen later. If there exists a circle $\gamma \in \CE_0$ such that $|\{ \gamma' \ |\ (\gamma, \gamma') \in \tau(\CE_0)\}| < c_1 N^{1/2}$
remove it from the collection and label the new refined collection as $\CE_1$. From this refined collection, if there exists a circle $\gamma$ such that $|\{\gamma'\ |\ (\gamma, \gamma') \in \tau(\CE_1)\}| < c_1 N^{1/2}$ we remove it and label the remaining collection as $\CE_2$.
After repeating this process $M$ times until there are no more circles $\gamma$ that satisfy $|\{\gamma'\ |\ (\gamma, \gamma') \in \tau(\CE_M)\}| < c_1 N^{1/2}$, at each step removing a circle that contributes only a small number of tangencies, we attain a collection $\CE_M$.
 We claim that $|\tau(\CE_M)| \gtrsim N^{\alpha}$, and that $|\CE_M| \gtrsim N^{\alpha/2}$.

For the first claim, observe that at each step $i$ we are reducing $\tau(\CE_i)$ by at most $c_1 N^{\alpha -1}$.  Thus,
\begin{align*}
    |\tau(\CE_M)| &\geq |\tau(\CE_0)| - M c_1 N^{\alpha -1} \\
    &> c_0 N^{\alpha} -  M c_1 N^{\alpha -1} \\ 
    &> c_0 N^{\alpha} - c_1 N^{\alpha} \\
    |\tau(\CE_M)| &> \frac{c_0}{2} N^{\alpha}. \qquad (\text{by choosing } c_1 = c_0/2 )
\end{align*}
We now provide a lower bound on the cardinality of our refined set $\CE_M$. We have the trivial inequality $|\tau(\CE_M)| \leq |\CE_M|^{2}$. 
Combining this with the result above, we attain $|\CE_M| \gtrsim |N|^{\alpha/2}$.
\end{proof}

We can now prove the main theorem using a strategy is similar to that of Chapter \ref*{chap:kakeya}. We assume that there are $\gtrsim N^{3/2}$ points of incidence and argue by contradiciton.
We will find a polynomial which has a zero set containing
many points of incidence between our curves, and then use the structure of the problem to claim that this polynomial's zero set must then also contain all the circles in the collection.
We do this to achieve a contradiction as the degree of our polynomial will be too low to contain this many circles in its zero set.
\begin{proof}[Proof of Theorem \ref{thm:circle-tangencies}]
    Given an arbitrary collection of circles $\CE$ with $\gtrsim N^{3/2}$ tangencies, Lemma \ref{lem:uniform-refine} with $\alpha = \frac{3}{2}$ 
    allows us to reduce to a collection
     $\Gamma \subset \CE$ where each circle is tangent to  $\gtrsim N^{1/2}$ other circles. 
    After applying a small rotation, we can assume that the tangent line at each point of tangency does not point vertically in the $y$-direction.
    Let $\beta (\Gamma) = \{ \beta(\gamma) \ | \ \gamma \in \Gamma \}$, where $\beta$ is the lifting transform defined earlier.
    Recall from Lemma \ref{lem:beta-lift} that two circles $\gamma_1$ and $\gamma_2$ are tangent if and only if $\beta(\gamma_1) \cap \beta(\gamma_2) \neq \emptyset$.

    Suppose $(x,y,z) \in \beta(\gamma_1) \cap \beta(\gamma_2) $ for some $\gamma_1 \neq \gamma_2$. 
    Then we claim that $$(0,0,1) \in \text{span} \left( T_{(x,y,z) }\beta (\gamma_1), T_{(x,y,z) }\beta (\gamma_2)\right).$$
    In other words, at the intersection of $\beta(\gamma_1)$ and $\beta(\gamma_2)$ their tangent vectors span a vertical subspace of $\RR^3$. 
    We can establish this by examining a parameterisation of $\gamma_1$ and $\gamma_2$ in the neighbourhood of $(x,y)$.
    Define $f_i (t), \ i \in \{1,2\}$ such that $(x+t, f_i(t))$ is a parameterisation of $\gamma_i$ in the neighbourhood of $(x,y)$ for all $t$ in a small neighbourhood of 0. In particular $f_i(0) = y$. Since $(x+t, f_i(t))$ satisfies the equation of a circle $\gamma_i$ we have the relation \todo{enough or more?}
    \[
        (x+t - x_{\gamma_i})^2 +  (f_i(t) - y_{\gamma_i})^2 = r_{\gamma_i}^2.
    \]
    Taking the first derivative at $t=0$ and rearranging we attain
    \[
        \frac{df_i}{dt}(0) = -\frac{x- x_{\gamma_i}}{y- y_{\gamma_i}}.
    \]
    Since $\gamma_1$ is tangent to $\gamma_2$ at $(x,y)$ the slopes of their tangent lines coincide at that point so $\frac{df_1}{dt}(0) = \frac{df_2}{dt}(0)$. Now taking the second derivative at $t=0$ and rearranging
    \[
        \frac{d^2f_i}{dt^2}(0) = -\frac{(y- y_{\gamma_i})^2 + (x- x_{\gamma_i})^2}{(y- y_{\gamma_i})^3} = -\frac{r^2_{\gamma_i}}{(y- y_{\gamma_i})^3} .
    \]
    Since $\gamma_1$ and $\gamma_2$ are distinct circles, $\frac{d^2f_1}{dt^2}(0) \neq \frac{d^2f_2}{dt^2}(0)$. (In the case $r_{\gamma_1} = r_{\gamma_2}$, notice that the tangency must be external and hence $\text{sgn}(y-y_{\gamma_1})^3\neq \text{sgn}(y-y_{\gamma_2})^3$)



    In the neighbourhood of $(x,y,z)$, $\beta(\gamma_i)$ is parametrised by $\left(t,f_i (t) ,\frac{df_1}{dt}(t) \right)$ as the slope of the tangent to the circle is given by $\frac{df_1}{dt}(t)$.
     It follows that the tangent vector
    $\left(1,\frac{df_i}{dt}(0), \frac{d^2f_i}{dt^2} (0) \right)$ generates the vertical space $T_{(x,y,z)} \beta(\gamma_i)$. Thus 
    \begin{align} (0,0,1) &\in \text{span}\left( \left(1,\frac{df_1}{dt}(0), \frac{d^2f_1}{dt^2} (0) \right) - \left(1,\frac{df_2}{dt}(0), \frac{d^2f_2}{dt^2} (0) \right) \right) \nonumber
    \\ &\subset \text{span} \left( T_{(x,y,z) }\beta (\gamma_1), T_{(x,y,z) }\beta (\gamma_2)\right). \label{eqn:z-span}
    \end{align}

    We will now interpolate $\sim N^{3/2}$ incidence points with a minimal polynomial whose degree we can control, and show that if this contains too many intersections it must also contain the curves. 
    Then due to the tangent vectors spanning the $z$-axis at the incidence points we will be able to achieve a contradiction.

    For each $\beta(\gamma) \in \beta(\Gamma)$ define a collection $K_\gamma$ as a set of $\sim N^{1/2}$ incidence points of $\beta(\gamma)$. Then $|\cup_\Gamma K_\gamma| \lesssim N^{3/2}$.
    Let $P \in \RR[x,y,z]$ be a non-zero polynomial of minimal degree that vanishes on all the incidence points in $\cup_\Gamma K_\gamma$. 
    This polynomial interpolates $\lesssim N^{3/2}$ points in $\RR^3$, so by Lemma \ref{lem:paramcounting}
    the degree of $P$ is $\lesssim \left(N^{3/2}\right)^{\frac{1}{3}} =  N^{1/2}$.

    Due to our refinement each $\gamma \in \Gamma$ is 
    tangent to $\gtrsim N^{1/2}$ circles and each of these tangencies occurs at a distinct point by the non-degeneracy condition.   
    Hence we have that $P$ vanishes at $\gtrsim N^{1/2}$ points of each curve in $\beta (\Gamma)$.
    By \hyperref[lem:Bezout]{Bézout's Lemma} (Lemma \ref{lem:Bezout}) we have that $P$ vanishes on all curves in $\beta(\Gamma)$ as
    \[
    \deg (P) \deg (\beta(\gamma))  \lesssim \# \{ P \cap \beta(\gamma)\}  
    \]
    for suitable choice of implicit constants. 

    By Equation \ref{eqn:z-span}, if $(x,y,z)$ is a point where two curves from $\beta (\Gamma)$ intersect, then $\partial_z P (x,y,z)  =0 $. 
    Thus by the same Bézout argument $\partial_z P$ is also a polynomial which vanishes on all curves in $\beta(\Gamma)$. 

    Since $P$ was a non-zero polynomial of minimal degree that vanishes on all points of incidence in $\beta (\Gamma)$, we must conclude 
    $\partial_z P \equiv$ 0 and hence we must have that $P(X,Y,Z) = Q(X,Y)$ for some $Q \in \RR[X,Y]$ with  $ \deg Q = \deg P \lesssim N^{1/2}$. 
    But this implies that there must be at least of the $N^{3/4}$ circles in our original collection which are contained in $Z(Q)$.
    This is a contradiction, as $Q$ has degree $\sim N^{1/2}$ whereas $\cup \gamma$ has degree $ \gtrsim N^{3/4}$.
    Thus $\beta(\Gamma)$ has $\lesssim N^{3/2}$ curve-curve incidences, so we can conclude that $\Gamma$ has $\lesssim N^{3/2}$ tangencies.
\end{proof}

\section[New Proof via Polynomial Partitioning]{A New Proof via Polynomial Partitioning for Varieties}
In this section we shall present a new proof of Theorem \ref{thm:circle-tangencies}. 
There are a few key differences between the two methods of proof. In the following proof, 
it is no longer required to ensure the uniform distribution of tangent points among each circle in our collection, so we will not
have to use Lemma \ref{lem:uniform-refine}.
Secondly, in a similar fashion to the proof of the Szemerédi-Trotter Theorem (Theorem \ref{thm:S-T}) in the previous chapter we will partition $\RR^3$ using a polynomial of controlled degree and leverage our trivial bound (Lemma \ref{lem:trivial-circle-bound}) in each cell.
Slight care will be needed to deal with intersections of curves in the zero set, which we do via a recursive  argument.

We begin by introducing the main tool for this proof, 
an extension of the \hyperref[thm:PolyPartioning]{polynomial partitioning} theorem (Theorem \ref{thm:PolyPartioning}) to algebraic varieties instead of just points (which are 0-dimensional varieties). 

\todo{add def of variety}
\todo{rephrase closer to \ref{thm:PolyPartioning}}
\begin{lemma}[Polynomial Partitioning for Algebraic Varieties]
    If $\Gamma$ is a finite set of $k$-dimensional varieties in $\RR^n$ and $D$ any degree, there exists a non-zero polynomial $P$ of degree at
    most $D$ such that each disjoint open set of $\RR^n \backslash Z(P)$ intersects $\lesssim D^{k-n} |\Gamma|$ varieties of $\Gamma$.\label{lem:poly-part-var}
\end{lemma}
We shall not prove this here. A proof of this result can be found as Theorem 0.3 in \cite{guth2015polypartvar}, the original paper of Guth presenting the result. The proof is in the same spirit as the proof Theorem \ref{thm:PolyPartioning} given in Chapter \ref{chap:trotter}, however some additional difficulties present themselves and a correct proof requires some care. One should notice that the $k=0$ case above recovers Theorem \ref{thm:PolyPartioning}.

\begin{proof}[Proof of Theorem \ref{thm:circle-tangencies} by Polynomial Partitioning]
Relabel $\CE$ as $\Gamma$ for convenience. Again, we perform the lifting transform $\beta$ on each $\gamma \in \Gamma$. 
We have a collection of $N$ 1-dimensional varieties in $\RR^3$ upon which we use our polynomial partitioning lemma (Lemma \ref{lem:poly-part-var})
to find a polynomial $P$ such that each cell of $\RR^3 \backslash Z(P)$ intersects $\lesssim N D^{-2}$ varieties. 
$\RR^3 \backslash Z(P)$ partitions the space into $\sim D^3$ cells. 
Let us label the interior of each of these cells as $\Omega_i$ for $0 \leq i \lesssim D^3$, and further label the set of varieties in $\Gamma$ that intersect a given cell $\Omega_i$ as $\Gamma_i$.

We can now define the following complementary sets based on whether the variety is contained entirely in $Z(P)$:
\begin{align*}
    C_1 &= \{ \beta(\gamma) \ |\ \beta(\gamma) \not \subset Z(P) \},\\
    C_2 &= \{ \beta(\gamma) \ |\ \beta(\gamma)  \subset Z(P) \}.
\end{align*}
Notice here that $\beta(\Gamma) = C_1 \cup C_2$. Recalling the correspondence between curve-incidences and  circle-tangencies, we define the following
incidence sets for any subcollections $C$ and $C'$ of $\beta(\Gamma)$
\begin{align*}
    I(C, C') &= \{(\beta(\gamma), \beta(\gamma')) \ |  \ \beta(\gamma) \in C', \ \beta(\gamma') \in C, \ \beta(\gamma) \cap \beta(\gamma') \neq \emptyset \}.
\end{align*}
Hence we can express $|\tau(\Gamma)|$ as the sum
\[
    |\tau(\Gamma)| = |I(C_1, C_1)| + 2|I(C_1, C_2)| + |I(C_2, C_2)|.  
\]
We now proceed to bound the cardinality for each of these sets. For $I(C_1,C_1)$ will use our trivial bound (Lemma \ref{lem:trivial-circle-bound}) and Bézout's
Lemma (Lemma \ref{lem:Bezout}). Bounding the cardinality for $I(C_1,C_2)$ is again straightforward by Bézout's Lemma. 
The interesting case here is $I(C_2,C_2)$, where we will be forced to argue via a recursive argument.

We begin with $I(C_1,C_1)$, that is, the intersections that occur between varieties not entirely contained in the zero set. We proceed by partitioning $I(C_1,C_1)$ according to whether the incidence occurs within a cell or on $Z(P)$, explicitly 
\begin{align*}
    I(C_1,C_1) = &\{ (\beta(\gamma), \beta(\gamma')) \ |  \ \beta(\gamma), \beta(\gamma') \in C_1, \ \beta(\gamma) \cap \beta(\gamma') \in \RR^3 \backslash Z(P)\} \\
                & \cup \{ (\beta(\gamma), \beta(\gamma')) \ |  \ \beta(\gamma), \beta(\gamma') \in C_1, \ \beta(\gamma) \cap \beta(\gamma') \in Z(P)\}.
\end{align*}
Hence we have
\begin{align*}
    |I(C_1,C_1)| &= \sum_{\beta(\gamma), \beta(\gamma') \in C_1} \OO[\beta(\gamma) \cap \beta(\gamma') \in \RR^3 \backslash Z(P)] + 
                  \sum_{\beta(\gamma), \beta(\gamma') \in C_1} \OO[\beta(\gamma) \cap \beta(\gamma') \in Z(P)] \\
                 &= \sum_{\beta(\gamma), \beta(\gamma') \in C_1} \sum_i \OO[\beta(\gamma) \cap \beta(\gamma') \in \Omega_i] +
                 \sum_{\beta(\gamma), \beta(\gamma') \in C_1} \OO[\beta(\gamma) \cap \beta(\gamma') \in Z(P)]\\
                 &= \sum_i \sum_{\beta(\gamma), \beta(\gamma') \in \Gamma_i}  \OO[\beta(\gamma) \cap \beta(\gamma') \in \Omega_i] +
                  \sum_{\beta(\gamma), \beta(\gamma') \in C_1} \OO[\beta(\gamma) \cap \beta(\gamma') \in Z(P)].
\end{align*}
Using our trivial bound (Lemma \ref{lem:trivial-circle-bound}) and the fact that there are $\lesssim ND^{-2}$ varieties intersecting a given cell we attain
\begin{align*}
                 &\lesssim \sum_i \left(\frac{N}{D^2}\right)^2 + \sum_{\beta(\gamma), \beta(\gamma') \in C_1} \OO[\beta(\gamma) \cap \beta(\gamma') \in Z(P)] \\
                 &\sim D^3 N^2 D^{-4} + \sum_{\beta(\gamma)\in C_1} \sum_{\beta(\gamma') \in C_1} \OO[\beta(\gamma) \cap \beta(\gamma') \in Z(P)].
                 \intertext{For each fixed $\beta(\gamma) \in C_1$ we see by Bézout's Lemma (Lemma \ref{lem:Bezout}) that $\beta(\gamma)$ can intersect $Z(P)$ in at most $\lesssim D$ points. Since all the intersections in the latter sum occur in $Z(P)$, we have that each $\beta(\gamma)$ contributes at most $\lesssim D$ to the sum, and summing over all possible $\beta(\gamma)'s \in C_1$ we obtain}
                 &\lesssim  N^2 D^{-1} + \sum_{\beta(\gamma)\in C_1} D \\
                 &\lesssim N^2D^{-1} + ND.
\end{align*}


The argument for $I(C_1,C_2)$ is similar to the one above, however there are now no intersections happening inside the cells by the definition of $C_2$, hence we have
\begin{align*}
    |I(C_1,C_2)| &= \sum_{\beta(\gamma)\in C_1} \sum_{\beta(\gamma') \in C_2} \OO[\beta(\gamma) \cap \beta(\gamma') \in Z(P)]. \\
    \intertext{Again thanks to Bézout's Lemma (Lemma \ref{lem:Bezout}) we have that the latter sum is $\lesssim D$. Hence}
    |I(C_1,C_2)|& \lesssim \sum_{\beta(\gamma)\in C_1} D \leq ND.
\end{align*}

Finally we need to handle the incidences between curves contained entirely in the zero-set, $I(C_2,C_2)$.
This is a slightly more technical undertaking than the previous two cases.
We proceed by considering the class of polynomials whose zero set contains every curve in $C_2$, formally defined as
\[
    \mathcal P_0 = \{  R(X,Y,Z) \in \RR[X,Y,Z] \ | \ \forall \beta(\gamma) \in C_2, \ \beta(\gamma) \subset Z(R) \}.  
\]

$\mathcal P_0$ is non-empty, as by the definition of $C_2$ it at least contains the partitioning polynomial $P$. Let us take the polynomial in $\mathcal P_0$ of minimal degree and label it $P_0$. 
As seen in the proof in Section \ref{sect:Zahl-Proof}, since the curves are entirely contained in $Z(P_0)$, and at a point of incidence we have $\partial_z P_0 =0$, we conclude $Z(\partial_z P_0)$ contains all the points in the set $I(C_2,C_2)$.  Either $\partial_z P_0 \equiv 0$ or it is not. 

In the first case, we must have that $P_0(X,Y,Z) = Q(X,Y)$ for some polynomial $Q \in \RR[X,Y]$. 
$Z(Q)$ contains all the circles $\gamma$ such that $\beta (\gamma) \in C_2$ (to see this, notice that $\beta(\gamma)$ restricted to $\RR^2$ is just $\gamma$). 
By Bézout's Lemma, $Z(Q)$ contains at most $\lesssim \deg \partial_z P_0$ circles, hence we can trivially bound the incidences here by $|I(C_2,C_2)| \leq (\deg \partial_z P_0 )^2 \lesssim D^2$.
In this case there is nothing left to do as we have bounded $|I(C_2, C_2)|$, so the recursion stops.

In the second case, we must have that $ \deg \partial_z P_0 < \deg P_0 \leq D$. Since $P_0$ was of minimal degree in $\mathcal P_0$, $Z(\partial_z P)$ cannot contain all the curves of $C_2$.
Therefore we can split the collection into a complementary pair of sets  
\begin{align*}
    C_1^{(1)} &= \{ \beta(\gamma) \in C_2 \ | \  \beta(\gamma) \not \subset Z(\partial_z P_0)  \}, \\
    C_2^{(1)} &=  \{ \beta(\gamma) \in C_2 \ | \  \beta(\gamma) \subset Z(\partial_z P_0)  \}.
\end{align*}

Using these sets we can express 
\begin{align*}
    |I(C_2,C_2)| &= \left| I\left(C_1^{(1)}, C_1^{(1)}\right) \right|+2 \left|I\left(C_1^{(1)}, C_2^{(1)}\right)\right| +\left| I\left(C_2^{(1)}, C_2^{(1)}\right)\right| 
    \intertext{Notice that since all points of incidence belong to $Z(\partial_z P_0)$, the set $I\left(C_1^{(1)}, C_1^{(1)}\right)$ is empty. Hence}
    &= \left|I\left(C_1^{(1)}, C_2\right)\right| +\left| I\left(C_2^{(1)}, C_2^{(1)}\right)\right|. 
\end{align*}
It follows from Bézout's Lemma via the same argument given before that $\left|  I\left(C_1^{(1)}, C_2\right)\right| \lesssim \left|C_1^{(1)} \right| D$, and we will now repeat this process to bound $\left|I \left(C_2^{(1)}, C_2^{(1)}\right)\right|.$

We now move into the next step of the recursion. Now we consider the class of polynomials
\[
    \mathcal P_1 = \{ R(X,Y,Z) \in \RR[X,Y,Z] \ | \ \forall \beta(\gamma) \in C_2^{(1)}, \beta(\gamma) \subset Z(R) \}. 
\]

$\mathcal P_1$ is non-empty, as by the definition of $C_2^{(1)}$ it at least contains the polynomial $\partial_z P_0$ (and hence the minimal degree polynomial in $\mathcal{P}_1$ is of degree at most $D-1$). Let us take the polynomial in $\mathcal P_1$ of minimal degree and label it $P_1$. Again we have that $Z(\partial_z P_1)$ contains all the points of $I\left(C_2^{(1)},C_2^{(1)}\right)$.  Again by the law of the excluded middle we have either $\partial_z P_1 \equiv 0$ or it is not. 

In the first case, again we must have that $P_1(X,Y,Z) = Q(X,Y)$ for some polynomial $Q \in \RR[X,Y]$. $Z(Q)$ contains all the circles $\gamma$ such that $\beta (\gamma) \in C_2$, of which there are  $\lesssim \deg \partial_z P_0$. Hence we can trivially bound the incidences here by \[\left|I\left(C_2^{(1)},C_2^{(1)}\right)\right| \leq (\deg \partial_z P_1 )^2 \lesssim D^2.\] If this is the case the recursion ends. 

In the second case, we must have that $ \deg \partial_z P_1 < \deg P_1 < \deg \partial_z P_0 \lesssim D$. Again, since $P_1$ was of minimal degree in $\mathcal P_1$, $Z(\partial_z P_1)$ cannot contain all the curves
of $C_2$. Hence we now define a new complementary pair of sets as 
\begin{align*}
    C_1^{(2)} &= \{ \beta(\gamma) \in C_2^{(1)} \ | \  \beta(\gamma) \not \subset Z(\partial_z P_1)  \}, \\
    C_2^{(2)} &=  \{ \beta(\gamma) \in C_2^{(1)} \ | \  \beta(\gamma) \subset Z(\partial_z P_1)  \}.
\end{align*}

Using these sets we can express 
\begin{align*}
    \left|I\left(C_2^{(1)},C_2^{(1)}\right)\right| &= \left| I\left(C_1^{(2)}, C_1^{(2)}\right) \right|+ \left|I\left(C_1^{(2)}, C_2^{(2)}\right)\right| +\left| I\left(C_2^{(2)}, C_2^{(2)}\right)\right| \\
    &= \left|I\left(C_1^{(2)}, C_2^{(1)}\right)\right| +\left| I\left(C_2^{(2)}, C_2^{(2)}\right)\right|. 
\end{align*}
We have that $\left|  I\left(C_1^{(2)}, C_2^{(1)}\right)\right| \lesssim \left|C_1^{(2)} \right| D$. Again, we can continue the recursion to bound $\left|I \left(C_2^{(2)}, C_2^{(2)}\right)\right|.$

In the $i$-th step of our recursion, we define
\[
    \mathcal P_i  = \{ R(X,Y,Z) \in \RR[X,Y,Z] \ | \ \forall \beta(\gamma) \in C_2^{(i)}, \beta(\gamma) \subset Z(R) \}. 
\]
This will always be non-empty as by the definition of $C_2^{(i)}$, it always contains at least the polynomial $\partial_z P_{i-1}$. The polynomial of minimal degree in $\mathcal P_i$, denoted $P_i$, is of strictly smaller degree than $P_{i-1}$. Hence our recursion must end after at most $D$ steps. At each step, either $\partial_z P_i \equiv 0$ or it is not. 

If $\partial_z P_i \equiv 0$, we can bound the contribution of this step by 
\[\left|I\left(C_2^{(i)},C_2^{(i)}\right)\right| \lesssim D^2\] 
and our recursion ends.

In the second case, we again define the complementary sets
\begin{align*}
    C_1^{(i+1)} &= \{ \beta(\gamma) \in C_2^{(i)} \ | \  \beta(\gamma) \not \subset Z(\partial_z P_i)  \}, \\
    C_2^{(i+1)} &=  \{ \beta(\gamma) \in C_2^{(i)} \ | \  \beta(\gamma) \subset Z(\partial_z P_i)  \}.
\end{align*}
As before, these will lead to a contribution of $\lesssim \left|  C_1^{(i+1)}\right| D + \left|I \left( C_2^{(i+1)},C_2^{(i+1)} \right) \right|$. The process continues on the latter term.

Let us collect the contributions from each step of our recursion. At the terminal step of the process, say step $J$,  we collect a term of $\lesssim D^2$. Collecting the terms preceeding this, we are able to bound the contribution by 
\[
    \sum_{i=0}^{J-1} \left| C_1^{(i+1)} \right| D \lesssim N D
\]
by using the face that all $ C_1^{(i+1)}$ are disjoint subsets of $C_2$ and that $C_2$ contains at most $N$ curves, we can conclude that
\[
  |I(C_2,C_2)| \lesssim D^2 + ND.
\]


Adding together our bounds we achieve
\begin{align*}
    |I(C_1,C_1)|+2|I(C_1,C_2)|+|I(C_2,C_2)| &\lesssim N^2D^{-1} + ND + ND + ND + D^2 \\
    &\sim N^2D^{-1} + ND + D^2.
\end{align*}
We now optimise $D$ by setting $N^2D^{-1} \sim ND$ which gives $D \sim N^{1/2}$ and hence our bounds become

\begin{align*}
    |I(C_1,C_1)|+|I(C_1,C_2)|+|I(C_2,C_2)| &\lesssim N^2N^{-1/2} + NN^{1/2} + (N^{1/2})^2 \sim N^{3/2}.
\end{align*}
\end{proof}
\begin{remark}
    Since polynomial partitioning exploits the topology of $\RR^n$ in an essential manner, it would have been reasonable to expect an improvement over the exponent 3/2 via the argument above. Alas, this was not the case.
\end{remark}

\section[The Case of Sphere Tangencies in $\RR^{3}$]{Considerations for the Case of Sphere Tangencies in $\RR^{3}$}\todo{capitalisation?}
Considering the analogous problem in $\RR^{3}$, two new difficulties emerge. 
Firstly, our current non-degeneracy condition is not sufficient as we have new degenerate cases to consider where the trivial bound of $O(N^2)$ is achieved. 
\begin{example}[Degenerate case in $\RR^{3}$, \cite{ZahlSpheres}]
    Consider the two collections of $N$ spheres illustrated in Figure \ref{fig:degen-3d}. Under the union of these collections, each sphere in the first collection is tangent to $\sim N$ spheres in the second collection at $N$ distinct points. 
    Hence the union is a collection of $2N$ spheres that produces $N^2$ tangencies. Note that for each sphere in the first collection, all of its tangencies with the second collection occur along a circle on the surface of the sphere.
    \begin{figure}[h]
        \centering 
        \includegraphics[width=0.8\textwidth]{images/degenr3.png}
        \caption{Two collections of $N$ spheres in $\RR^3$, whose union satisfies the non-degeneracy condition the earlier section yet achieves $N^2$ tangencies. \label{fig:degen-3d}}
        \end{figure}
\end{example}
Due to cases like these a new condition is required. 
A natural choice given this example would seem to be restricting our collections to those which do not contain too many tangencies 
along any one low-degree algebraic curve on the surface of a sphere. 

Secondly, our simple Bézout lemma is no longer enough. An analogous lifting of a sphere $\gamma \subset \RR^3$ is given by

\begin{align*}
    \beta(\gamma) = \Biggl \{ (x,y,z,v,w) \in \RR^5 \  \big | \ &(x-x_\gamma)^2 + (y-y_\gamma)^2 + (z-z_\gamma)^2 = r_\gamma^2, \\
     &v = - \frac{x-x_\gamma}{z-z_\gamma}, \ w = - \frac{y-y_\gamma}{z-z_\gamma} \qquad  \Biggr \} .
\end{align*}


Notice that $\beta(\gamma)$ is a two-dimensional algebraic surface in $\RR^5$. An interpolating polynomial's zero set $Z(P)$ such as the one used in Ellenberg-Solymosi-Zahl's proof would be a 4-dimensional surface in $\RR^5$.
These two surfaces could intersect along a 1-dimensional variety (i.e. at infinitely many points), hence we can no longer argue that $\beta(\gamma) \subset Z(P)$ due to $Z(P)$ containing too many points of $\beta(\gamma)$.

The best example of a lower bound for such tangencies is produced in a similar fashion to the circles case (Example \ref{ex:circle-lower-bound}). 
In place of the Szemerédi-Trotter theorem we appeal to a result of \cite{Guibas1990}, later refined by the authors in \cite{Sharir2007} to remove a subpolynomial term.\footnote{The original bound in \cite{Guibas1990} was multiplied by a factor of $|\ES|^\delta|\Pi|^\delta$ for some $\delta>0$.}
\begin{theorem}[\cite{Guibas1990} , \cite{Sharir2007}]
    Let $\ES$ be a finite set of points in $\RR^3$ and $\Pi$ a finite set of collinear planes in $\RR^3$ such that no three are collinear. \label{thm:point-plane-incidences}
    Then the incidence set $I(\ES, \Pi)$ obeys
    \[
        |I(\ES, \Pi)| \lesssim |\ES|^{4/5}|\Pi|^{3/5} + |\ES| + |\Pi|.
    \] 
\end{theorem}
The above bound is indeed tight as shown by a construction in \cite{Knauer2003}.
We now construct our lower bound utilising geometric inversion in $\RR^3$ analogously to Example \ref{ex:circle-lower-bound}.

\begin{example}
    Let $\ES$, $\Pi$ be collections of $N$ points and $N$ non-collinear planes respectively such that the Theorem \ref{thm:point-plane-incidences} is asymptotically tight. 
    In particular, they determine $\sim N^{7/5}$ point-plane incidences. 
    Let $C_1$ denote the collection of unit spheres centred at the points of $\ES$. 
    Let $C_2$ denote the collection of planes obtained by translating each plane $\pi \in \Pi$ one unit in the $\pi^{\perp}$ direction.
    If $(p,\pi) \in \ES \times \Pi$ is a point-plane incidence from our original collection, then the corresponding sphere-plane pair will be tangent.
    Performing an inversion transform about a point that does not lie in any of the circles or planes, our collection $\CE = C_1 \cup C_2$ becomes a 
    collection of $2N$ spheres that determine $N^{3/2}$ tangencies.
 \end{example}

 In \cite{ellenberg2016new}, the authors conjectured a bound for higher dimensional spheres:
 \begin{conjecture}
     Let $\CE$ be a collection of $(d-1)$-spheres in $\RR^d$. Then the number of distinct points of tangencies is $\lesssim |\CE|^{\frac{2d-1}{d}}.$
 \end{conjecture}

This is a somewhat weak conjecture, as the evidence to form it is simply the natural extension of the $\beta$-lifting technique into higher dimensions. \todo{rewrite sentence}
As such, there is no evidence to suggest that it should be tight. 
