\chapter{The Circle Tangency Counting Problem}
\todo{Kakeya proof with a twist}
Here we shall discuss the problem of counting the number of tangencies in a suitably non-degenerate collection of circles. We say two circles are tangent if their intersection contains a single point.\todo{is this stupid?}
The set of unordered pairs of circles in a collection $\CE$ which are mutually tangent are called the tangencies of the collection and is denoted $\tau(\CE)$. 
\section{Trivial Bounds}
\begin{theorem}[Trivial Bound]
    Let $\CE$ be an arbitrary finite collection of circles. Then the number of tangencies $|\tau(\CE)|$ is bounded as follows:
    \[
        |\tau(\CE)| \leq |\CE|^2
    \]
\end{theorem}
Stated for arbitrary collections of circles the problem is not particularly interesting as the above bound turns out to be asymotopically tight. 
\begin{example}
Denote a circle centred at $(x,y)$ with radius $r$ as $\gamma (x,y,r)$. \todo{notation}
Consider the following collection of $N$ circles:
\begin{align*}
    C_0 &= \gamma\left(0,0,1\right) \\
    C_1 &= \gamma\left(\frac{1}{2},0,\frac{1}{2}\right) \\
    C_2 &= \gamma\left(\frac{3}{4},0, \frac{1}{4}\right) \\
    &\vdots \\
    C_N &= \gamma\left(1- \frac{1}{2^N}, 0, \frac{1}{2^N}\right)
\end{align*}
Each circle in our collection $\CE = \cup_{0 \leq i \leq N} C_i$ is tangent to $N$ other circles at the point (1,0). 
Hence $|\tau(\CE)| \sim N^2 \sim |\CE|^2$. 
\todo{diagram}
\end{example}

Instead we shall look at collections of circles that satisfy a single non-degeneracy condition. We consider collections of circles 
such that no three are mutually tangent at a common point. The best known examples leverage our tight \hyperref[thm:S-T]{Szemerédi-Trotter Theorem} examples:
\begin{example}
    Let $\ES$, $\LL$ be collections of $N$ points and $N$ lines respectively such that the \hyperref[thm:S-T]{Szemerédi-Trotter Theorem}'s bound is tight. 
    In particular, they determine $\sim N^{4/3}$ point line incidences. 
    Let $C_1$ denote the collection of unit circles centred at the points of $\ES$. 
    Let $C_2$ denote the collection of lines obtained by translating each line $\ell in \LL$ one unit in the $\ell^{\perp}$ direction.
    If $(p,\ell) \in \ES \times \LL$ is a point-line incidence from our original collection, then the corresponding circle-line pair will be tangent.
    Performing an inversion transform about a point that does not lie in any of the circles or lines, our collection $\CE = C_1 \cup C_2$ becomes a 
    collection of $2N$ circles that determine $N^{4/3}$ tangencies.
 \end{example}

We shall now present a bound from a recent paper of Ellenberg-Solymosi-Zahl which uses the polynomial method.\cite{ellenberg2016new}

\begin{theorem}
    Given a finite collection $\CE$ of $N$ circles in the plane such that no three are tangent at a common point, 
    then the number of tangencies $|\tau(\CE)|$ obeys:
    \[
        |\tau(\CE)| \lesssim N^{3/2}
    \]
\end{theorem}

It is not obvious how to apply the polynomial method in the current formulation. 
We will perform a lifting  of our circles into algebraic curves $\RR^3$, 
preserving tangencies between circles in the plane as intersections between their lifted curves in $\RR^3$. 
This transforms the problem from a tangency problem to an incidence problem, and reduces the required degree of polynomial needed to interpolate.

We shall now discuss this lifting in detail. Let $\gamma \subset \RR^2$ be a circle in the plane of radius $r_{\gamma}$ centred at $(x_\gamma, y_\gamma)$. 
We define the lifting transform of the circle as: \[
    \beta(\gamma) = \left \{ (x,y,z) \in \RR^3 \ | \ (x-x_\gamma)^2 + (y-y_\gamma)^2 = r_\gamma^2, \ z = - \frac{x-x_\gamma}{y-y_\gamma} \right \}.
\]  
Clearly $\beta$ is an algebraic curve, so now we shall examine the correspondence between mutually tangent circles. Notice that $z$ is defined as the slope of the tangent line
at the point $(x,y) \in \gamma$.

\begin{lemma}
    Let $\beta$ be the transform defined as above. 
    Then two circles $\gamma, \gamma' \subset \RR^2$ are tangent if and only if $\beta(\gamma) \cap \beta(\gamma') \neq \emptyset$.
    \label{lem:beta-lift}
\end{lemma}
\begin{proof}
If $\gamma$ and $\gamma'$ are tangent, then there exists a point $(x,y) \in \gamma \cap \gamma'$ so the first two coordinates of $\beta(\gamma)$ and $\beta(\gamma')$ agree. 
Further, at a point of tangency we have that the slope of the tangent lines coincide for $\gamma$ and $\gamma'$, so the final coordinate is also equal. Hence $\beta(\gamma) \cap \beta(\gamma') \neq \emptyset$.

In the other direction, assume there exists some $(x,y,z) \in \beta(\gamma) \cap \beta(\gamma')$. Clearly $(x,y) \in \gamma \cap \gamma'$, and again the slope of the tangent lines at this point are equal, hence we conclude
$\gamma$ is tangent to $\gamma'$.
\end{proof}
This one-to-one correspondence between tangencies in $\RR^2$ and incidences in $\RR^3$ is the key idea behind the proof. To use the polynomial method here, we will need to ensure our tangencies
are sufficiently uniformly distributed among our circles. We can refine our collection such that this is the case by the following lemma.

\begin{lemma}
    Let $\CE$ be a collection of $N$ circles and suppose that  $\tau(\CE) \gtrsim N^{\alpha}$. 
    Then we can refine our set such that every circle in $\CE' \subset \CE$ is tangent to $\gtrsim N^{\alpha -1}$ circles.
\end{lemma}

\begin{proof}
Let $\tau(\CE)$ be the set of tangencies of the circles in $\CE$. Let $\CE_0 = \CE$. Take a circle $\gamma \in \CE_0$ such that $|\{(\gamma, \gamma') \ |\ (\gamma, \gamma') \in \tau(\CE_0)\}| < c_1 N^{1/2}$
and discard it. We label our new refined collection as $\CE_1$. From this collection, again take a circle $\gamma$ such that $|\{(\gamma, \gamma') \ |\ (\gamma, \gamma') \in \tau(\CE_1)\}| < c_1 N^{1/2}$ and discard it. 

After repeating this process $M$ times until there are no more circles that satisfy our criteria, 
at each step removing a circle that contributes only a small number of tangencies, we attain a collection $\CE_M$.
 We claim that $\tau(\CE_M) \gtrsim N^{\alpha}$, and that $\CE_M \neq \emptyset$.

For the first claim, observe that at each step $i$ we are reducing $\tau(\CE_i)$ by at most $c_1 N^{\alpha -1}$.  Thus,
\begin{align*}
    |\tau(\CE_M)| &\geq |\tau(\CE_0)| - M c_1 N^{\alpha -1} \\
    &> c_0 N^{\alpha} -  M c_1 N^{\alpha -1} \\ 
    &> c_0 N^{\alpha} - c_1 N^{\alpha} \\
    |\tau(\CE_M)| &> \frac{c_0}{2} N^{\alpha}. \qquad (\text{by choosing } c_1 = c_0/2 )
\end{align*}
We must now check that we have not removed every circle from our collection.\todo{disagree with correction here, tangencies is not necessarily non-zero, why else get this lower bound?} We have the trivial inequality $|\tau(\CE_M)| \leq |\CE_M|^{2}$. 
Combining this with the result above, we attain $|\CE_M| \gtrsim |N|^{\alpha/2}$.
\end{proof}
In a similar fashion to our arguments in the Kakeya problem we will be arguing by contradiction, however here we will be arguing that if the zero set of a polynomial contains too many of a certain first type of object, 
it must contain many more of a second kind of object. 

We will need a result from Bezout which bounds the number of intersections between curves in $\RR^3$.

\begin{lemma}[Bezout's Theorem for Plane Algebraic Curves]
   Let $Z_1$ and $Z_2$ be the zero sets of two polynomials over $\RR[X,Y]$. Then the following holds:
   \[
       |\{ x | x \in Z_1 \cap Z_2 \}| \leq \deg Z_1 \deg Z_2
   \]
    \label{lem:Bezout}
\end{lemma}


We can now prove the main theorem. 
\begin{proof}
    Given an arbitrary collection of circles $\CE$ with $\gtrsim N^{3/2}$ tangencies, we can reduce to a collection $\Gamma$ where each circle is tangent to at least $\sim N^{1/2}$ other circles using the previous lemma. 
    After applying a small rotation, we can assume that the tangent line at each point of tangency does not point vertically in the $y$-direction.
    Let $\beta (\Gamma) = \{ \beta(\gamma) : \gamma \in \Gamma \}$, where $\beta$ is the lifting transform defined earlier,
    Recall from Lemma \ref{lem:beta-lift} that two circles $\gamma_1$ and $\gamma_2$ are tangent if and only if $\beta(\gamma_1) \cap \beta(\gamma_2) \neq \emptyset$.

    Suppose $(x,y,z) \in \beta(\gamma_1) \cap \beta(\gamma_2) $ for some $\gamma_1 \neq \gamma_2$. 
    Then $$(0,0,1) \in \text{span} \left( T_{(x,y,z) }\beta (\gamma_1), T_{(x,y,z) }\beta (\gamma_2)\right).$$
    In other words, at the intersection of $\beta(\gamma_1)$ and $\beta(\gamma_2)$ their tangent vectors span a vertical subspace of $\RR^3$. 
    We can establish this by examining a parameterisation of $\gamma_1$ and $\gamma_2$ in the neighbourhood of $(x,y)$.
    Define $f_i (t), \ i \in \{1,2\}$ such that $(t+x, f_i(t))$ is a parameterisation of $\gamma_i$ in the neighbourhood of $(x,y)$ for all $t$ in a small neighbourhood of 0. In particular $f_i(0) = y$.
    Since $\gamma_1$ is tangent to $\gamma_2$ at $(x,y)$, $\frac{df_1}{dt}(0) = \frac{df_2}{dt}(0)$. 
    Since $\gamma_1$ and $\gamma_2$ are distinct quadratic curves, $\frac{d^2f_1}{dt^2}(0) \neq \frac{d^2f_2}{dt^2}(0)$. \todo{this was intuitive, not anymore :(}
    In the neighbourhood of $(x,y,z)$, $\beta(\gamma_i)$ is parametrised by $\left(t,f_i (t) ,\frac{df_1}{dt}(t) \right)$ as the slope of the tangent to the circle is given by $\frac{df_1}{dt}(t)$.\todo{I'm not convinced in this explan.} 
     It follows that the tangent vector
    $\left(1,\frac{df_i}{dt}(0), \frac{d^2f_i}{dt^2} (0) \right)$ generates the vertical space $T_{(x,y,z)} \beta(\gamma_i)$. Thus 
    \begin{align*} (0,0,1) &\in \text{span}\left( \left(1,\frac{df_1}{dt}(0), \frac{d^2f_1}{dt^2} (0) \right) - \left(1,\frac{df_2}{dt}(0), \frac{d^2f_2}{dt^2} (0) \right) \right)
    \\ &\subset \text{span} \left( T_{(x,y,z) }\beta (\gamma_1), T_{(x,y,z) }\beta (\gamma_2)\right). 
    \end{align*}

    We will now interpolate all points of intersection with a polynomial of suitably low degree, and show that if this contains too many points it will lead to a contradiction.
    Let $P \in \RR[x,y,z]$ be a non-zero polynomial of minimal degree that vanishes on all intersections between the curves in $\beta (\Gamma)$. The degree of $P$ is $\sim N^{1/2}$. 
    By our result above, if $(x,y,z)$ is a point where two curves from $\beta (\Gamma)$ intersect, then $\partial_z P (x,y,z)  =0 $. Thus since each $\gamma \in \Gamma$ is 
    tangent to $\gtrsim N^{1/2}$, and each of these tangencies occur at a distinct point, we have that $\partial_z P$ vanishes at $\gtrsim N^{1/2}$ points on each curve in $\beta (\Gamma)$.
    By \hyperref[lem:Bezout]{Bézout's theorem} we have that $\partial_z P$ vanishes on all curves in $\Gamma$ as:
    \[
    \deg (\partial_z P) \deg (\gamma) \sim  (N^{1/2}) \gtrsim \# \{\partial_z P \cap \gamma\} \sim (N^{1/2}) .    
    \] Since $P$ was the non-zero polynomial of minimal degree that vanishes on all the curves in $\beta (\Gamma)$, we must conclude 
    $\partial_z P$= 0. We have then that $P(x,y,z) = Q(x,y)$ for some $Q \in \RR[x,y]$ with degree $\sim N^{1/2}$. 
    But this implies that each of the $N$ circles in $\Gamma$ must be in $Z(Q)$. This is a contradiction, as $Q$ has degree $\sim N^{1/2}$ whereas $\cup \gamma$ has degree $2N$.
    We conclude that $\Gamma$ has fewer than $N^{3/2}$ tangencies.
\end{proof}

