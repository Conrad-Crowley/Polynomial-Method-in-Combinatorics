\chapter{Introduction}
The following is a short exposition of the polynomial method in combinatorics. 
It is perhaps already misleading to call this the polynomial method 
as there are in fact a collection of polynomial methods that have applications in combinatorics, the first example of this being known since the 1990s
where Alon's proved a combinatorial version of Hibert's Nullstellensatz.\cite{alon1999combinatorial} We examine this result in detail
in Chapter \ref{chap:alon}. In 2008 Dvir produced a remarkably short resolution to the finite field analogue of the Kakeya Conjecture which provided
a new framework and enthusiasm for the polynomial method in combinatorial problems.\cite{2008DVIR} We will explore this proof in the next chapter. 

The most striking feature of the following proofs is that they leverage polynomials in problems which on the surface appear not to have anything to 
do with polynomials. Generally, extremal configurations of these problems tend to have a lot of algebraic structure and this is exactly what these methods exploit using polynomials. 



In Chapter \ref{chap:kakeya} we will use a low degree polynomial to interpolate a certain set of objects. We will then show that the zero set of the polynomial can not contain 
too many objects of this type due to the inherent structure of these objects. In Chapter \ref{chap:joints} we will $\dots$ \todo{do something indeed}.
Chapter \ref{chap:trotter} introduces the idea of using a polynomial to partition the plane into a degree-controlled number of cells each containing at most some uniform amount of points.\todo{weak}
Chapter \ref{chap:circle} will discuss a proof that first uses a similar strategy to Chapter \ref{chap:kakeya}, however in this proof we will be interpolating
objects of one type and then proving that the polynomial's zero set cannot contain too many objects of some second related type. 
We then provide a new proof of the same theorem which uses an extension of the polynomial partitioning seen in \ref{chap:trotter}, the key difference being that
instead of controlling the number of points in each cell we will control the number of varieties that intersect any given cell. Finally, in Chapter \ref{chap:alon} we will provide an example
of the polynomial method being used outside the context of incidence geometry.

\section{Why polynomials?}
\todo{this section}

Points to possibly mention here:
\begin{enumerate}
    \item choosing a polynomial has $D^n$ degrees of freedom [$\text{Poly}_D(\FF^n) \sim D^n$]
    \item polynomials behave rigidly on lines, having only $D$ degrees of freedom.
    \item ``parameter counting - vanishing lemma'' method. 
    \item Non constructive method of finding a polynomial is reminiscent of probabilistic method.\cite{GOW2020} \cite{GUTH2016}
    \item ill-formed thoughts that polynomials extract information about finite series analogous to generating functions and infinite series. 
\end{enumerate}

\section*{Notation}
We introduce some convenient notation here. We write that $A \lesssim_n B$ to mean that there exists some constant
$C(n)$ which depends on $n$ such that $A \leq C(n) B$. Further, we write that $A \sim_n B$ if $A \lesssim_n B$ and $B \lesssim_n A$.

We write $\text{Poly}_D (\KK^n)$ to represent the space of polynomials in $n$ variables with coefficients in $\KK$ and degree at most $D$.

The indicator function $\OO$ is defined on logical statements $X$ as follows:
\[
    \OO[X] = 
  \begin{cases}
      1 & \text{if X is true}, \\
      0 & \text{if X is false}.
  \end{cases}  
\]
For any function $f : \RR^n \to \RR$ let us denote the zero set of $f$ by $Z(f) = \{x \in \RR^n \ | \ f(x) = 0\}.$
We borrow from Computer Science the big O notation. For functions $f,g : \NN^+ \to \RR$ we write:
\begin{align*}
    f(N) &= O(g(N)) \iff \exists N_0, M \in \NN \text{ such that } f(n) \leq Mg(n) \forall n > N_0 \\
    f(N) &= \Omega(g(n)) \iff g(N) = O(f(N)).
\end{align*}
These can be thought of as asymtopic upper and lower bounds respectively.