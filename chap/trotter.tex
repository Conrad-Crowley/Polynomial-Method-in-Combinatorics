\chapter{Szemerédi-Trotter Theorem}

In this chapter we will study the application of the polynomial method to incidence geometry by proving a fundamental theorem in the field.
Incidence geometry is the study of possible intersection patterns of simple geometric objects, such as lines or low degree curves. 
We have already seen an incidence problem in the previous chapter on the Joints problem. 
By developing the powerful tool of polynomial partitioning we shall see the key role that the topology of $\RR$ can play in such problems, 
in contrast to the trivial topology of finite fields. 


\section{Background}
The Szemerédi-Trotter theorem is a fundamental theorem to the field of incidence geometry, originally proved by an involved cell decomposition argument
of Szemerédi and Trotter and later given a shorter proof using crossing numbers by Székely. 
\begin{theorem}[Szemerédi-Trotter]
    Let $\ES \subset\RR^2$ be a finite set of points and
    let $\LL \subset \RR^2$ be a finite set of lines. We define 
    \[I(\ES,\LL) = \{(p, \ell) \in \ES \times \LL \ | \ p \in \ell\}\] 
    to be the set of incidences between $\ES$ and $\LL$. 
   
    Then:
    \[
        |I(\ES, \LL)| \lesssim \left(|\ES||\LL|\right)^{2/3} + |\ES| + |\LL|
    \]
    \label{thm:S-T}
\end{theorem}

\section{The Trivial Bound}
In planar geometry, we have the following dual statements: two points determine a line and every pair of lines intersect in at most one point.
Using this we can prove the following bounds on $I(\ES,\LL)$:
\begin{theorem}[Trivial Bounds]
    For a set of points $\ES$ and lines $\LL$ we have
    \[I(\ES,\LL)| \leq |\ES|^2 + |\LL|. \]
    \[I(\ES,\LL)| \leq |\LL|^2 + |\ES|. \]
    \label{thm:trivial-ST-bounds}
\end{theorem}
\begin{proof}

To see this, count the lines that have at most one point in $P$ on them. These contribute at most $|\LL|$ incidences.
Every other line has at least two points in $\ES$. The total number of incidences on these lines is at most $|\ES|^2$
as otherwise there would exist a $p\in \ES$ that lies on over $|\ES|$ lines, and each of these lines would have an additional 
point on it. This would imply there are more that $|\ES|$ points, a contradiction. 

Interchanging the roles of $\LL$ and $\ES$ achieves the other bound. \todo{does it?}
\end{proof}

\begin{theorem}[Second Trivial Incidence Bound]
    $$I(\ES,\LL) \lesssim |\ES|\cdot |\LL|^{\frac{1}{2}} + |\LL|$$
    and
    $$I(\ES,\LL) \lesssim |\LL|\cdot |\ES|^{\frac{1}{2}} + |\ES|.$$
    \label{thm:pretty-trivial-ST-bounds}
\end{theorem}
\begin{proof}
    


We now bound the number of incidences. 

\begin{align*}
    |I(\ES,\LL)|^2 &= \left( \sum_{\ell \in \LL} \sum_{p \in \ES} \OO[p\in \ell] \right)^2 \\
    & \leq |\LL|\cdot \sum_{\ell \in \LL} \left( \sum_{p \in \ES} \OO[p\in \ell] \right)^2  \qquad \text{ (Cauchy-Schwarz on $\ell$)}\\
    &= |\LL|\cdot \sum_{p_1,p_2 \in \ES} \sum_{\ell \in \LL}   \OO[p_1\in \ell] \OO[p_2\in \ell]    \\
    & \leq |\LL|\cdot ( |I(\ES,\LL)| + |\ES|^2)\\ 
    &\leq |\LL|^2 + 2|\LL|\cdot |\ES|^2 \qquad \text{ (Using Theorem \ref{thm:trivial-ST-bounds}) }\\ 
\end{align*}    \todo{extra step needed here - split up sum and do triple thing}
This implies
$$I(\ES,\LL) \lesssim |\ES|\cdot |\LL|^{\frac{1}{2}} + |L|.$$ 
Repeating the above proof interchanging the roles $\ES$ and $\LL$ achieves the other bound.
\end{proof}


\section{Examples}
We can not improve beyond our second trivial bounds in a finite field $\FF^2$. 
\begin{example}[Finite Fields]
Consider the set of points $\ES=\FF^2$ and let $\LL$ be the set of all lines 
in $\FF^2$. Every line contains exactly $|\FF|$ many points of $\ES$, so we have $|\FF|^3$ incidences. 
So both sides of the \hyperref[thm:pretty-trivial-ST-bounds]{second trivial bound} are comparable:
$$
I(\ES,\LL) = |\FF|^3 \sim (|\FF|^2)(|\FF|^2)^{1/2} + |\FF|^2.
$$
\end{example}
In contrast, the following examples seem to be the best possible over $\RR$. We will later prove that these are the tight case of the \hyperref[thm:S-T]{Szemerédi–Trotter Theorem}.
We define a line in $\RR^2$ as follows:
\[
    \ell_{m,c} = \{(x,y) \in \RR^2 \ | \ y = mx+c\}.     
\]
\begin{example}
    Consider the following collections in $\RR^2$:
    \begin{align*}
    \ES &= \{ (a,b) \in \ZZ^2 \ | \ a\in [1,N], b \in [1,2N^2] \} \\
    \LL &= \{ \ell_{m,c} \in \RR^2 \ | \ m,c\in \ZZ, m\in [1,N], c \in [1,N^2] \} \end{align*}
    The collection $\ES$ contains $2N^3$ points and $\LL$ contains $N^3$ lines. Every line in $\LL$ contains $N$ points in $\ES$ as for each $x \in [1,N]$  the y-coordinate of $\ell_{m,c}$, $ mx+c$, gives a different integer in $[1,2N^2]$.
    Hence there are $N^4$ incidences. Both sides of the \hyperref[thm:S-T]{Szemerédi-Trotter} inequality are comparable as 
    $$ I(\ES,\LL) = N^4 \sim (N^3)^{\frac{2}{3}} (N^3)^{\frac{2}{3}} \sim |\ES|^{2/3} |\LL|^{2/3}$$
\end{example}
\todo{diagram?}
\begin{example}
    Let $N \gg 1$ be a large even integer and let $1<R \ll N$ be another integer. Consider the collections in $\RR^2$:
    \begin{align*}
        \ES &= \{ (a,b) \in \ZZ^2 \ | \ (a,b) \in [-N/2,N/2] \times [-N/2,N/2] \} \\
        \LL &= \{ \ell \ | \ \ell \text{ contains between R and 2R points of } \ES \} \end{align*}
    We begin by estimating how many lines pass through a given point of the regular grid $\ES$.
    Let $\ell \in \LL$ and $p \in \ES$. The closest point $p' \in \ES$ such that $p \neq p'$ and $p' \in \ell$ must lie
    in a square centred at $p$ of side length $\sim N/R$. This follows from the fact that there are at least $R$\todo{check here} points of $\ES$ in $\ell$ and hence the
    projections of these points to the axes can be separated by at most $\sim N/R$. Taking each possible combination of these we can conclude that there are $\lesssim N^2 / R^2$ in $\LL$ through a
    given point $p$.

    We now claim that there are $\gtrsim N^2 / R^2$ distinct such lines. We need only consider the points in the upper right quadrant of $\ES$ as the problem is symmetrical.
    Further, we restrict ourselves to considering lines with slopes $m$ satisfying $\frac{1}{2} < m < 2$. For such a line to contain $R$  points
    of $\ES$ we require $m= \frac{l}{k} \in \QQ$ with $\gcd(l,k) =1$ and $l,k \in \left[\frac{N}{2R}, \frac{N}{R} \right]$. There are $\gtrsim N^2/R^2$ pairs, as the proportion of pairs that share a factor of 2 is $\frac{1}{2}^2$ and the proportion of pairs that share a factor 
    of 3 is $\frac{1}{3}^2$, and in general the proportion that shares a factor of $k$ is $\frac{1}{k}^2$. We have that $\sum_{k>1} \frac{1}{k}^2 < \frac{2}{3} < 1$ and hence there are $\gtrsim N^2 /R^2$ distinct
    lines in $\LL$ through each point. Taking account of what we have shown: 
    \begin{align*}
        |\ES| &\sim N^2 \\ 
        |\LL| &\sim |\ES| \frac{N^2}{R^2} \frac{1}{R} \sim \frac{N^4}{R^3}\\
        |I(\ES, \LL)| &\sim |\ES| \frac{N^2}{R^2} \sim \frac{N^4}{R^2}\\
    \end{align*}
    we can see that both sides of the \hyperref[thm:S-T]{Szemerédi-Trotter} inequality are comparable:
    \[
    |I(\ES, \LL)| \sim \frac{N^4}{R^2} \sim (N^2)^{\frac{2}{3}} \left(\frac{N^4}{R^3}\right)^{\frac{2}{3}} \sim |\ES|^{\frac{2}{3}}|\LL|^{\frac{2}{3}}
    \]
\end{example}
\todo{add diagram (!)}



\section{Ham Sandwich Theorems}
The above examples suggest that the topology of $\RR$ plays a key role in this incidence problem. We shall now introduce the method of polynomial partitioning,
which can be seen as the topological analogy to the \hyperref[kaklem2]{vanishing lemma} we used in the previous chapters. 

Let $\SP^n$ denote the unit $n$-sphere in $\RR^{n+1}$. 
\begin{theorem}[Borsuk-Ulam]
    A map $\phi$ is said to be antipodal if it obeys $\phi (-x) = -\phi(x)$ for all $x$ in its domain. Suppose $\phi: \SP^N \to \RR^N$ is a continuous antipodal mapping. 
    Then the image of $\phi$ contains 0. 
    \label{thm:Borsuk-Ulam}
\end{theorem}
The proof of this result is long and beyond the scope of this paper. We refer interested readers to a beautiful geometric proof in chapter 2 of Matousek's book \textit{Using the Borsuk-Ulam theorem}.\cite{matouvsek2003using}

Let us now define some useful notation going forward. 
\begin{definition}[Bisection of a Set]
A function $f :\RR^n \to \RR$ is said to bisect an open set $U$ with volume $\text{Vol}(U) < \infty$ if:
\[
    \text{Vol}\{x \in U \ | \ f(x) > 0 \} = \text{Vol}\{x \in U \ | \ f(x) < 0 \} = \frac{1}{2} \text{Vol}(U).
\]
Analogously, a function $f$ is said to bisect a finite set $S$ if both:
\[
  |\{x \in S \ | \ f(x) > 0\}| \leq \frac{|S|}{2}   
\]
and 
\[
  |\{x \in S \ | \ f(x) < 0\}| \leq \frac{|S|}{2}.
\]

\end{definition}


\begin{theorem}[General Ham Sandwich Theorem]
Let V be a finite dimensional vector space of continuous functions $f:\RR^n \to \RR$ such that for any non-zero function $f$,
 $Z(f)$ has zero Lebesgue measure. 
Let $U_1,U_2,\dots,U_N \subset \RR^n$ be finite volume open sets with $N< \dim V$. 

Then there exists a non-zero function $f \in V$ that bisects each $U_i$. \label{thm:GenHamSand}
\end{theorem}
\begin{proof}
Define the functions $\{\phi_i\}_{i=1}^N$, $\phi_i: V\backslash \{0\} \to \RR$ by
\[
\phi_i(f) = \text{Vol}(\{x\in U_i \ | \ f(x) > 0 \}) - \text{Vol}(\{x\in U_i \ |\ f(x) > 0 \})
\]
Since $Z(f)$ has measure zero, it is easy to see that $\phi_i(f) = 0$ if and only if $f$ bisects $U_i$. 
Notice also that $\phi_i(-f) = -\phi_i(f)$, hence $\phi_i$ is antipodal. 

We now show each $\phi_i(F)$ is continuous.
It is enough to show that if $U$ is a finite volume open set, then the measure of $\{x\in U\ |\ f(x)>0\}$ depends continuously on $f\in V \backslash \{0\}$.

Suppose $f_n \to f$ in $V$ for some $f,f_n \in V \backslash \{0\}$. $f_n$ converges to $f$ in the topology of $V$, 
so it follows it must converge pointwise. Pick any $\varepsilon >0$. 
By Egorov's theorem, we can find a subset $E\subset U$ so that $f_n \to f$ uniformly pointwise on $U \backslash E$, and $m(E)< \varepsilon$.
By hypothesis, $m(Z(f)) =0$ and $m(U) < \infty$. 
Since the Lebesgue measure is continuous we can choose $\delta$ such that $m\left(\{x\in U\ |\ |f(x)|<\delta\}\right) < \varepsilon$.

Now we choose $n$ sufficiently large that $|f_n (x) - f(x)| < \delta$ on $U\backslash E$. Then we have \[|m\left(\{x\in U\ |\ f_n(x)>0\}\right) - m\left(\{x\in U\ |\ f(x)>0\}\right)| < 2 \varepsilon.\] Since $\varepsilon$ was arbitrary each $\phi_i$ is continuous.
\todo{is this just cnty of everything? - ask marco}


We now combine each $\phi_i$ into the map $\phi : V\backslash \{0\} \to \RR^N$. Since $\dim V > N$, select a subspace $U < V$ such that $\dim U = N+1$. 
Now choose an isomorphism of $U$ with $\RR^{N+1}$, and think of $\SP^N$ as a subset of $U$.
Now the map $\phi: \SP^N \to \RR^N$ is antipodal and continuous. 
By the Borsuk-Ulam theorem, there exists an $f\in \SP^N \subset V\backslash \{0\}$ such that $\phi(f) = 0$.
\end{proof}

\begin{corollary}[Finite Ham Sandwich Theorem]
    Let $S_1, \dots , S_N$ be finite sets in $\RR^n$ and let $D$ be such that $N < {{D+n}\choose{n}}$. 
    Then there exists a non-zero $P\in \text{Poly}_D (\RR^n)$ that bisects each $S_i$.  \label{thm:FiniteHamSandwich} 
\end{corollary}

\begin{proof}
    For each $\delta>0$, define $U_{i, \delta}$ to be the union of $\delta-balls$ centred at the points of $S_i$. 
    By Theorem \ref{thm:GenHamSand}, we can find a non-zero $P_{\delta}$ with degree less than $D$ that bisects each $U_{i, \delta}$. 
    By rescaling we can assume \todo{abusing notation, also the norm is l1?}$P_{\delta} \in \SP^N \subset \text{Poly}_D (\RR^n) \backslash \{0\}$.
    Since $\SP^N$ compact, we can find a sequence $\delta_m \to 0$ so that $P_{\delta_{m}}$ converges to $P$ in $\SP^N$.
    Since the coefficients of $P_{\delta_{m}}$ converge to $P$, $P_{\delta_{m}}$ converges to $P$ uniformly on compact sets.

    We claim $P$ bisects each $S_i$. 
    By contradiction, suppose $P>0$ on more than half the points of $S_i$, say on the points of $S_i^+$. 
    Choosing $\varepsilon$ sufficiently small, we can assume $P>0$ on the $\varepsilon$-ball around each point of $S_i^+$.
    Further, we can choose $\varepsilon$ such that each $\varepsilon$-ball is disjoint. 
    Since $P_{\delta_{m}}$ converges uniformly, we can find $m$ sufficiently large such that $P_{\delta_{m}}>0$ 
    on the $\varepsilon$-ball around each point of $S_i^+$.
    By making $m$ large, we can also arrange that $\delta_m < \varepsilon$.
    Thus $P_{\delta_{m}} > 0$ on more than half the points of $U_{i, \delta_{m}}$.
\end{proof}

\begin{theorem}[Polynomial Partitioning]
    For any $n$ there exists a constant $c(n)$ such that if $S$ is a finite subset of $\RR^n$ and $D$ is any degree, then there exists
    a polynomial $P$ of degree $D$ such that $\RR \backslash Z(P)$ is a disjoint union of $\lesssim D^n$ open sets $O_i$ each containing
    $\lesssim_n |S|D^{-n}$ points. \label{thm:PolyPartioning}
 \end{theorem}
 
 \begin{proof}
 The main idea of this proof is the repeated application of the \hyperref[thm:FiniteHamSandwich]{Finite Ham Sandwich Theorem}. We begin by finding a polynomial $P_1$ of degree 1 that bisects $S$. This partitions $\RR \backslash Z(P_1)$
 into two disjoint open sets according to the sign of $P_1$, $P_1^+$ and $P_1^-$, each containing at most $|S|/2$ points.
 We then bisect both of these sets using another polynomial $P_2$.
 There are four sign conditions on $P_1$ and $P_2$, these being the four possible intersections of the sets $P_1^{\pm}$ and $P_2^{\pm}$,
 and the subset for each sign condition contains at most $|S|/4$ points of $S$. 
 Continuing this process to define polynomials $P_3, P_4, \dots$, where the polynomial $P_j$ simultaneously bisects $2^{j-1}$ finite sets. 
 By the \hyperref[thm:FiniteHamSandwich]{Finite Ham Sandwich Theorem}, each $P_j$ can have a degree
 $\lesssim 2^{j/n}$. 
 Repeating this procedure $J$ times, and defining $P = \prod_{i=1}^{J} P_i$, $\RR^n \backslash Z(P)$ is the disjoint union of $2^J$ open sets each containing $ \leq |S|2^{-J}$
 points of $S$. Now we choose $D$ such that $\deg(P) < D$ which is equivalent to $\sum_{j=0}^J c(n) 2^{j/n} \leq D$. But $\sum_{j=0}^J 2^{j/n}$ is a geometric series so we can find $\deg (P) < D$ for $D \leq c(n) 2^{J/n}$. 
 The number of points in each $O_i$ is $\leq |S| 2^{-J} \leq c(n) |S| D^{-n}$
 \end{proof}

There is a crucial point to note about polynomial partitioning. 
The above theorem does not guarantee anything about the distribution of points between $Z(P)$ and its compliment.
This is made most clear looking at the extremal examples. If all points line in the compliment of $Z(P)$ then we have an optimal eqidistribution of points,
and can often use trivial bounds in a divide-and-conquer style argument. On the otherhand, in the case all points are contained in $Z(P)$ we have
many points in an algebraic surface of controlled degree, so we can try and use tools from algebraic geometry. Generally, there will be some points in both $Z(P)$ and its compliment,
which we need to deal with separately.


\section{Proof of the Szemerédi-Trotter Theorem}
We now can prove the \hyperref[thm:S-T]{Szemerédi-Trotter} theorem using polynomial partitioning.
\begin{proof}[Proof of the Szemerédi-Trotter Theorem]
Let $|\ES| = S$ and $|\LL| = L$.
We need only consider the case $S^{\frac{1}{2}} \leq L \leq S^2$, as otherwise the proof follows immediately from the lemma above. 
Let $D$ be a degree to be chosen later. By Theorem \ref{thm:PolyPartioning}, there
exists a polynomial $P$ of degree $D$ such that $\RR^2 \backslash Z(P)$ splits into $D^2$ components each having $\lesssim SD^{-2}$ points. 
Let $O_{i \in \Pi}$ denote these components and let $\ES_i = \ES \cap O_i$, $\LL_i$ denote the lines that intersect the interior of each $O_i$ respectively.
We define the following pairs of complimentary sets:
\begin{align*}
    \ES_c &=\{x \in \ES \ |\ x \not \in Z(p)\}\\
    \ES_z &=\{x \in \ES \ |\ x \in Z(p)\}\\
    \LL_c &=\{\ell \in \LL \ |\ \ell \not \subset Z(p)\}\\
    \LL_z &=\{\ell \in \LL \ |\ \ell \subset Z(p)\}
\end{align*} 
Note that $\ES = \ES_c \cup \ES_z$, $\LL = \LL_c \cup \LL_z$. We can now write our total line-point incidences as the following sum
$$I(\ES, \LL) = I(\ES_c, \LL) + I(\ES_z, \LL_z) + I(\ES_z, \LL_c).$$\todo{check equality?}
If a line $\ell$ is not contained entirely in $Z(P)$ then it can intersect $P$ at most $D$ times, 
 so each line intersects at most $D+1$ cells. Hence $\sum_{i \in \Pi} L_i \leq (D+1)L$. We begin by examining the $I(\ES_c, \LL)$ term:
\begin{align*}
I(\ES_c, \LL) &= \sum_{i \in \Pi} I(\ES_i, \LL_i)
\end{align*}
Using our \hyperref[thm:trivial-ST-bounds]{trivial bound} in each cell:
\begin{align*}
&\leq \sum_{i \in \Pi} \ES_i^2 + \sum_{i \in \Pi} \LL_i\\
&\lesssim LD + SD^{-2} \sum_{i \in \Pi} S_i\\ &\leq LD + S^2D^{-2}
\end{align*} 


The number of lines in $\LL_z$ is at most $D$. So we have by our \hyperref[thm:trivial-ST-bounds]{trivial bounds}:
$$I(\ES_z, \LL_z) \leq S + D^2.$$
Each line in $\LL_c$ has at most $D$ intersection points with $Z(P)$ so it has at most $D$ incidences with $\ES_z$. Hence:
 $$I(\ES_z, \LL_c) \leq LD.$$
Together we have now 
$$I(\ES, \LL) \lesssim LD + S^2D^{-2} +S + D^2.$$
We optimise $LD + S^2D^{-2}$ by choosing $D$ such that both terms comparable and hence $D \sim S^{\frac{2}{3}} L^{-\frac{1}{3}}$. 
From our restriction $S^{\frac{1}{2}} \leq L \leq S^2$ we have $S^{\frac{2}{3}} L^{-\frac{1}{3}} \geq 1$
and $D^2 \sim S^{\frac{4}{3}} L^{-\frac{2}{3}} \leq S$, so we achieve
$$I(\ES, \LL) \lesssim (SL)^{2/3} + S. $$

Considering the regime where $L > S^2$ and applying the \hyperref[thm:trivial-ST-bounds]{trivial bound} yields the full Szemerédi-Trotter inequality:
\[
    I(\ES, \LL) \lesssim (SL)^{2/3} + S + L. 
\]
\end{proof}

There are two key things to note about the above proof. First, the key role that the topology of $\RR$ plays. Topology is used in the proof of polynomial partitioning as
it relies on the Borsuk Ulam theorem. It is a worthwhile heuristic to develop that polynomial partitioning may be useful for incidence problems
where the best examples in a finite field (which is only equipped with the trivial topology) do not coincide with the best known examples over the reals.
Secondly, the above proof illustrates the surprising power of polynomial partitioning. We are able to use very trivial bounds in each cell to achieve
a tight overall bound. 
\todo{consider adding a digression on ST}
