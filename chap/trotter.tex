\chapter{Szemerédi–Trotter Theorem}

In this chapter we will study the application of the polynomial method to incidence geometry by proving a fundamental theorem in the field.
Incidence geometry is the study of possible intersection patterns of simple geometric objects, such as lines or low degree curves. 
We have already seen an incidence problem in the previous chapter on the Joints problem. 
By developing the powerful tool of polynomial partitioning we shall see the key role that the topology of $\RR$ plays in such problems, in contrast to the trivial topology of finite fields. 


\section{Background}



\begin{theorem}[Szemerédi–Trotter]
    Let $\ES \subset\RR^2$ be a finite set of points and
    let $\LL \subset \RR^2$ be a finite set of lines. We define 
    \[I(\ES,\LL) = \{(p, \ell) \in \ES \times \LL \ | \ p \in \ell\}\] 
    to be the set of incidences between $\ES$ and $\LL$. 
   
    Then:
    \[
        |I(\ES, \LL)| \lesssim \left(|\ES||\LL|\right)^{2/3} + |\ES| + |\LL|
    \]
    
    \label{thm:S-T}
\end{theorem}

\section{The Trivial Bound}
In planar geometry, we have the following dual statements: two points determine a line and every pair of lines intersect in at most one point.
Using this we can prove the following bounds on $I(\ES,\LL)$:
\begin{theorem}[Trivial Incidence Bounds]
    $$I(\ES,\LL) \lesssim |\ES|\cdot |\LL|^{\frac{1}{2}} + |\LL|$$
    and
    $$I(\ES,\LL) \lesssim |\LL|\cdot |\ES|^{\frac{1}{2}} + |\ES|.$$
    \label{thm:trivial-ST-bounds}
\end{theorem}
\begin{proof}
    

We have that 
\eqq{I(\ES,\LL)| \leq |\ES|^2 + |\LL|. \label{eq:trivial-trivial-ST}}
To see this, count the lines that have at most one point in $P$ on them. These contribute at most $|\LL|$ incidences.
Every other line has at least two points in $\ES$. The total number of incidences on these lines is at most $|\ES|^2$
as otherwise there would exist a $p\in \ES$ that lies on over $|\ES|$ lines, and each of these lines would have an additional 
point on it. This would imply there are more that $|\ES|$ points, a contradiction. 

We now bound the number of incidences. 

\begin{align*}
    |I(\ES,\LL)|^2 &= \left( \sum_{\ell \in \LL} \sum_{p \in \ES} 1_{p\in \ell} \right)^2 \\
    & \leq |\LL|\cdot \sum_{\ell \in \LL} \left( \sum_{p \in \ES} 1_{p\in \ell} \right)^2  \qquad \text{ (Cauchy-Schwarz on $\ell$)}\\
    &= |\LL|\cdot \sum_{p_1,p_2 \in \ES} \sum_{\ell \in \LL}   1_{p_1\in \ell} 1_{p_2\in \ell}    \\
    & \leq |\LL|\cdot ( |I(\ES,\LL)| + |\ES|^2)\\
    &\leq |\LL|^2 + 2|\LL|\cdot |\ES|^2 \qquad \text{ (Using \ref{eq:trivial-trivial-ST}) }\\ 
\end{align*}    
This implies
$$I(\ES,\LL) \lesssim |\ES|\cdot |\LL|^{\frac{1}{2}} + |L|.$$ 
Repeating the above proof interchanging the roles $\ES$ and $\LL$ achieves the other bound.
\end{proof}


\section{Examples}
We can not improve beyond trivial bounds in a finite field $\FF^2$. Unlike the joints problem, where the results coincided for finite fields and the reals,
this suggests that the topology of $\RR$ plays a special role. 
\begin{example}[Finite Fields]
Consider the set of points $\ES=\FF^2$ and lines 
$\LL = \FF^2$. Every line contains exactly $|\FF|$ many points in $\FF^2$ \todo{show this}, so we have $|\FF|^3$ incidences. 
So both sides of the \hyperref[thm:trivial-ST-bounds]{trivial bounds} are comparable:
$$
I(\ES,\LL) = |\FF|^3 \sim (|\FF|^2)(|\FF|^2)^{1/2} + |\FF|^2.
$$
\end{example}
In contrast, the following examples seem to be the best possible over $\RR$. We will later prove that these are the tight case of the \hyperref[thm:S-T]{Szemerédi–Trotter Theorem}.
We define a line in $\RR^2$ as follows:
\[
    \ell_{m,c} = \{(x,y) \in \RR^2 \ | \ y = mx+c\}.     
\]
\begin{example}
    Consider the following collections in $\RR^2$:
    \begin{align*}
    \ES &= \{ (a,b) \in \ZZ^2 \ | \ a\in [1,N], b \in [1,2N^2] \} \\
    \LL &= \{ \ell_{m,c} \in \RR^2 \ | \ m,c\in \ZZ, m\in [1,N], c \in [1,N^2] \} \end{align*}
    The collection $\ES$ contains $2N^3$ points and $\LL$ contains $N^3$ lines. Every line in $\LL$ contains $N$ points in $\ES$ so there are
    $N^4$ incidences. Both sides of the \hyperref[thm:S-T]{Szemerédi-Trotter} inequality are comparable as 
    $$ I(\ES,\LL) = N^4 \sim (N^3)^{\frac{2}{3}} (N^3)^{\frac{2}{3}} \sim P^{2/3} L^{2/3}$$
\end{example}
\todo{diagram?}
\begin{example}
    $R$ \& $2R$ example
\end{example}
\todo{add this example}



\section{Ham Sandwich Theorems}
\begin{theorem}[General Ham Sandwich Theorem]
Let V be a vector space of continuous functions on $\RR^n$. Let $U_1,U_2,\dots,U_N \subset \RR^n$ be finite volume open sets with $N< \dim V$. For any function $f \in V\backslash \{0\}$, suppose $Z(f)$ has Lebesgue measure zero.

Then there exists a function $f \in V\backslash \{0\}$ that bisects each $U_i$. \label{thm:GenHamSand}
\end{theorem}
\begin{proof}
Define the functions $\{\phi_i\}_{i=1}^N$, $\phi_i: V\backslash \{0\} \to \RR$ by
\[
\phi_i(F) = \text{Vol}(\{x\in U_i | F(x) > 0 \}) - \text{Vol}(\{x\in U_i | F(x) > 0 \})
\]
We notice that each $\phi_i(F) = 0$ if and only if $F$ bisects $U_i$. Notice also that $\phi_i(-F) = -\phi_i(F)$, hence $\phi_i$ is antipodal. We now show each $\phi_i(F)$ is continuous.
It is enough to show that is $U$ is a finite volume open set, that the measure of $\{x\in U| f(x)>0\}$ depends continuously on $f\in V \backslash \{0\}$.

Suppose $f_n \to f$ in $V$ for some $f,f_n \in V \backslash \{0\}$. $f_n$ converges to $f$ in the topology of $V$, so it follows it must converge pointwise. Pick any $\varepsilon >0$. We can find a subset $E\subset U$ so that $f_n \to f$ uniformly pointwise on $U \backslash E$, and $m(E)< \varepsilon$.
By hypothesis, $m(Z(f)) =0$ and $m(U) < \infty$. Hence we can choose $\delta$ such that $m\left(\{x\in U| |f(x)|<\delta\}\right) < \varepsilon$.

Now we choose $n$ sufficiently large such that $|f_n (x) - f(x)| < \delta$ on $U\backslash E$. Then we have \[|m\left(\{x\in U| f_n(x)>0\}\right) - m\left(\{x\in U| f(x)>0\}\right)| < 2 \varepsilon.\] Since $\varepsilon$ was arbitrary each $\phi_i$ is continuous.

We now combine each $\phi_i$ into the map $\phi : V\backslash \{0\} \to \RR^N$. Since $\dim V > N$, let $\dim V = N+1$. Now choose an isomorphism of $V$ with $\RR^{N+1}$, and think of $S^N$ as a subset of $V$.
Now the map $\phi: S^N \to \RR^N$ is antipodal and continuous. By the Borsuk-Ulam theorem, there exists an $F\in S^N \subset V\backslash \{0\}$ such that $\phi(F) = 0$.
\end{proof}

\begin{corollary}[Finite Ham Sandwich Theorem]
    Let $S_1, \dots , S_N$ be finite sets in $\RR^n$ with $N < {{D+n}\choose{n}} = \text{Poly}_D (\RR^n)$. Then there exists a non-zero $P\in \text{Poly}_D (\RR^n)$ that bisects each $S_i$.  \label{thm:FiniteHamSandwich} 
\end{corollary}

\begin{proof}
    For each $\delta>0$, define $U_{i, \delta}$ to be the union of $\delta-balls$ centred at the points of $S_i$. By Theorem \ref{thm:GenHamSand}, we can find a non-zero $P_{\delta}$ with degree less than $D$ that bisects each $U_{i, \delta}$. By rescaling we can assume $P_{\delta} \in S^N \subset \text{Poly}_D (\RR^n) \backslash \{0\}$.
    Since $S^N$ compact, we can find a sequence $\delta_m \to 0$ so that $P_{\delta_{m}}$ converges to $P$ in $S^N$. Since the coefficients of $P_{\delta_{m}}$ converge to $P$, $P_{\delta_{m}}$ converges to $P$ uniformly on compact sets.

    We claim $P$ bisects each $S_i$. By contradiction, suppose $P>0$ on more than half the points of $S_i$, say on the points of $S_i^+$. By choosing $\varepsilon$ sufficiently small, we can assume $P>\varepsilon$ on the $\varepsilon$-ball around each point of $S_i^+$.
    Further, we can choose $\varepsilon$ such that each $\varepsilon$-ball is disjoint. 

    Since $P_{\delta_{m}}$ converges uniformly, we can find $m$ sufficiently large such that $P_{\delta_{m}}>0$ on the $\varepsilon$-ball around each point of $S_i^+$. By making $m$ large, we can also arrange that $\delta_n < \varepsilon$.
    Thus $P_{\delta_{m}} > 0$ on more than half the points of $U_{i, \delta_{m}}$.
\end{proof}
\section{Proof of Szemerédi–Trotter Theorem}
\begin{theorem}[Polynomial Partitioning]
   For any $n$ there exists a constant $c(n)$ such that if $S$ is a finite subset of $\RR^n$ and $D$ is any degree, then there exists
   a polynomial $P$ of degree $D$ such that $\RR \backslash Z(P)$ is a disjoint union of $\lesssim D^n$ open sets $O_i$ each containing
   $\leq c(n) |S|D^{-n}$ points. \label{thm:PolyPartioning}
\end{theorem}

\begin{proof}
We repeatedly apply Corollary \ref{thm:FiniteHamSandwich}. We begin by finding a polynomial $P_1$ of degree 1 that bisects $S$. We partition $\RR \backslash Z(P_1)$
into two disjoint open sets according to the sign of $P_1$, each containing at most $|S|/2$ points. We then bisect both of these sets using another polynomial $P_2$. There are four sign conditions on $P_1$ and $P_2$, and the subset for
each sign condition contains at most $|S|/4$ points of $S$. Continuing this process to define polynomials $P_3, P_4, \dots$, the polynomial $P_j$ bisects $2^{j-1}$ finite sets. By Corollary \ref{thm:FiniteHamSandwich}, we can find $P_j$ with degree
$\lesssim 2^{j/n}$. $\RR \backslash Z (P_1 \cdot P_2 \cdot \dots P_j)$ is the disjoint union of $2^j$ open sets each containing $\leq |S| 2^{-j}$ points. 
Repeating this procedure $J$ times, and defining $P = \prod_{i=1}^{J} P_i$, $\RR^n \backslash Z(P)$ is the disjoint union of $2^J$ open sets each containing $ \leq |S|2^{-J}$
points of $S$. Now we choose $D$ such that $\deg(P) < D$ which is equivalent to $\sum_{j=0}^J c(n) 2^{j/n} \leq D$. But $\sum_{j=0}^J 2^{j/n}$ is a geometric series so we can find $\deg (P) < D$ for $D \leq c(n) 2^{J/n}$. 
The number of points in each $O_i$ is $\leq |S| 2^{-J} \leq c(n) |S| D^{-n}$
\end{proof}

\begin{lemma} [Trivial Bounds]
    By double counting, we have the following trivial estimates:
    $$
    I(\ES, \LL) \leq L^2 + S,
    $$
    $$
    I(\ES, \LL) \leq S^2 + L.
    $$
    
    
\end{lemma}
\begin{proof}
We need only consider the case $S^{\frac{1}{2}} \leq L \leq S^2$, as otherwise the proof follows immediately from the lemma above. Let $D$ be a degree to be chosen later. By Theorem \ref{thm:PolyPartioning}, there
exists a polynomial $P$ of degree $D$ such that each component of $\RR^2 \backslash Z(P)$ has $\lesssim SD^{-2}$ points. Let $O_i$ denote these components and $\ES_i = \ES \cap O_i$, $\LL_i = \LL \cap O_i$. 
Note that $\ES = \ES_c \cup \ES_z$, $\LL = \LL_c \cup \LL_z$, where $\ES_z, \LL_z$ are the set of points and lines in $Z(P)$ respectively.
$$I(\ES, \LL) \leq I(\ES_c, \LL) + I(\ES_z, \LL_z) + I(\ES_z, \LL_c)$$
If a line $\ell \notin Z(P)$ then it can intersect $P$ at most $D$ times, and so each line intersects at most $D+1$ cells. Hence $\sum L_i \leq (D+1)L$.
\begin{align*}
    I(\ES_c, \LL) &= \sum_i I(\ES_i, \LL_i) \leq \sum_i \ES_i^2 + \sum_i \LL_i\\
    &\lesssim LD + SD^{-2} \sum_i S_i \leq LD + S^2D^{-2}
\end{align*} 
We also have by our lemma
$$I(\ES_z, \LL_z) \leq S + D^2$$
and finally $$I(\ES_z, \LL_c) \leq LD.$$
Together we have now 
$$I(\ES, \LL) \lesssim LD + S^2D^{-2} +S + D^2.$$
We optimise $D$ in $LD + S^2D^{-2}$ by making both terms comparable and hence $D \sim S^{\frac{2}{3}} L^{-\frac{1}{3}}$. From our restriction $S^{\frac{1}{2}} \leq L \leq S^2$ we have $S^{\frac{2}{3}} L^{-\frac{1}{3}} \geq 1$
and $D^2 \sim S^{\frac{4}{3}} L^{-\frac{2}{3}} \leq S$ so we achieve
$$I(\ES, \LL) \lesssim (SL)^{2/3} + S $$
\end{proof}