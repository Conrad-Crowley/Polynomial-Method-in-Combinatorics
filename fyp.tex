\documentclass[a4paper,11pt]{report}
%\usepackage[none]{hyphenat}

%\usepackage[T1]{fontenc}
%\usepackage[bitstream-charter]{mathdesign}

%\usepackage[usenames,dvipsnames,pdf]{pstricks}
%\usepackage[crop=off]{auto-pst-pdf}
\usepackage{makeidx}
\usepackage{url}
\usepackage[square,sort,comma,numbers]{natbib}
\usepackage{framed}
\usepackage[toc,page]{appendix}
\usepackage{graphicx}
\usepackage{longtable}
\usepackage{subfig}
\usepackage{amsfonts}
\usepackage{amsmath}
\usepackage{amsthm}
\usepackage{amssymb}
\usepackage{listings}
\usepackage{tikz}
\usepackage{tcolorbox}
\usepackage{float}
\usepackage{multirow}
%indicator 1
\usepackage{bbm}

% Fancy letters
\usepackage{mathrsfs}


\usepackage[noend,ruled,noline,linesnumbered,algochapter]{algorithm2e}
\usepackage[linktoc=page]{hyperref}
\usepackage[margin=3cm]{geometry}
\usepackage{setspace} 

\usepackage{fancyhdr}

% add todo package
\usepackage[colorinlistoftodos]{todonotes}

%add hyperref package
\usepackage{hyperref}

\pagestyle{fancy}

\rhead{\nouppercase\leftmark}
\lhead{\nouppercase\rightmark}

\makeindex

\onehalfspacing 

\renewcommand*\contentsname{Table of Contents}

% Proof, Corollary, and Lemma indexing
\newtheorem{theorem}{Theorem}[section]
\newtheorem{corollary}{Corollary}[theorem]
\newtheorem{lemma}[theorem]{Lemma}
\newtheorem*{remark}{Remark}
\newtheorem{example}{Example}[chapter]
\theoremstyle{definition}
\newtheorem{definition}{Definition}[section]
\newtheorem{conjecture}{Conjecture}

% New Commands 
\newcommand{\thm}[2]{\begin{theorem}[#1] #2 \end{theorem} }
\newcommand{\prf}[1]{\begin{proof} #1 \end{proof}}
\newcommand{\eq}[1]{\begin{equation*} #1\end{equation*}}
\newcommand{\eqq}[1]{\begin{equation} #1\end{equation}}
\newcommand{\lem}[1]{\begin{lemma} #1 \end{lemma}}

\newcommand{\CC}{\mathbb C}
% Fancy C
\newcommand{\CE}{\mathcal C}
\newcommand{\FF}{\mathbb F}
% Special F_p 
\newcommand{\FP}{\mathbb{ F}_p}
\newcommand{\KK}{\mathbb K}
% Fancy L
\newcommand{\LL}{\mathcal L}

\newcommand{\NN}{\mathbb N}
% indicatior 1
\newcommand{\OO}{\mathbbm 1}
%Fancy P
\newcommand{\PP}{\mathcal P}
\newcommand{\QQ}{\mathbb Q}
\newcommand{\RR}{\mathbb R}
\newcommand{\SP}{\mathbb S}

% Fancy S
\newcommand{\ES}{\mathcal S}
\newcommand{\ZZ}{\mathbb Z}

%indicator 
\newcommand{\1}[1]{\mathds{1}_{[#1]}}

% Polynomial space of degree <- where did this go?

% headheight to make the problems shut up for once.
\setlength{\headheight}{13.6pt}
\addtolength{\topmargin}{-1.6pt}

\begin{document}

\pagenumbering{roman}

%title page
\begin{titlepage}


    \begin{center}
    \line(1,0){430}\\
    \vspace{5pt}
    %{ \Huge \bf Heuristically Guided\\ Local Search for \\ Protein Structure Prediction \\}
    {\Huge \bf On Polynomial Methods in \\ \vspace{5pt} Combinatorics}
    \line(1,0){430}
    
    \end{center}
    
    \vfill
    
    \begin{center}

    \end{center}
    
    \vfill
    \begin{center}
    {\large \bf  Conrad Crowley }\\
    { \large \bf 118316041}\\
    
    
    \end{center}

    \begin{center}
        {\large \bf  Supervisor: Dr. Marco Vitturi }\\
        { \large \bf Second Reader: Dr. Andrei Mustata}\\
        
        \end{center}
    
    
    \vfill
    
    
    \begin{center}
    \end{center}
    
    \vfill
    \begin{center}
    {Final Year Project 2022}
    \end{center}
    
    \vfill
        
    \begin{center}
    \includegraphics[scale=0.2]{./images/ucc}
    \end{center}
    
    
    
    \end{titlepage}
    

%\clearpage
%\thispagestyle{empty}
%\mbox{}
\clearpage

%\thispagestyle{empty}
%Dedication
%I would like to dedicate this thesis to my loving family ...
    



%\clearpage
%\thispagestyle{empty}
%\mbox{}
%\clearpage
%\thispagestyle{empty}
\pagenumbering{roman}
\setcounter{page}{1}

%\input{0.1.abstract.tex}

\clearpage
\phantomsection
\tableofcontents
\renewcommand{\contentsname}{Table of Contents}
\addcontentsline{toc}{chapter}{Table of Contents}

\clearpage
\phantomsection
\addcontentsline{toc}{chapter}{List of Figures}
\listoffigures

%\clearpage
%\phantomsection
%\addcontentsline{toc}{chapter}{List of Tables}
%\listoftables



\clearpage
\pagenumbering{arabic}
\setcounter{page}{1}

\renewcommand\bibname{References}

\clearpage
\phantomsection


\chapter{Introduction}
The following is a short exposition of polynomial methods in combinatorics. Polynomial methods are a collection of techniques which use polynomial interpolation and rigidity properties
of polynomials to control the size of collections of objects with a certain structure. 
The first example of this technique was presented in the 1990s in \cite{alon1999combinatorial},  which we examine in detail
in Chapter \ref{chap:alon}. The modern conception of the polynomial method was pioneered by Dvir in 2008 (See \cite{2008DVIR}), where he produced a remarkably short resolution to the finite field analogue of the Kakeya Conjecture which provided
a new framework and enthusiasm for the polynomial method in combinatorial problems. We will explore this proof in the next chapter. 

The most striking feature of the following proofs is that they leverage certain properties of polynomials in problems which on the surface appear not to have anything to 
do with polynomials. Generally, extremal configurations of these problems tend to admit a lot of algebraic structure and this is exactly what these methods exploit using polynomials. 


\section{Why Polynomials?}
It is perhaps wise to discuss here what features of polynomials make them particularly powerful when dealing with problems in Combinatorics. 
Polynomials are perhaps some of the simplest of mathematical objects, as once we define a field they are simply a combination of the addition and multiplication operation between elements. 
It is not immediately obvious why such simple objects may prove to be so useful.

There are two key properties of polynomials that this collection of methods exploit.
Firstly, we use the fact that there are roughly $\sim D^n$ coefficients of a polynomial in $n$ variables of degree at most $D$ (See Lemma \ref{lem:paramcounting}).  This is utilised 
in an essential manner when we try to find polynomials that contain objects in their zero set. We can contain a set of size $M$ in the zero set of
a polynomial with degree at most $O(M^{1/n}$). In other words, we have a lot of flexibility in choosing a polynomial.
Secondly, and in sharp contrast, polynomials behave extremely rigidly when restricted to lines. We mean by this that the zero set of a polynomial of degree $D$ can intersect 
a line in at most $D$ points if the line is not contained within said zero set. The gap between this flexibility of choosing a polynomial and rigidity of restricting to lines provides
us with a surprisingly powerful technique. 

Another striking thing about the method is the non-constructive manner in which the polynomials are usually used. 
We often cannot explicitly find a satisfactory polynomial to use for our purposes, instead we opt to use arguments from Linear Algebra to establish the
existence of a polynomial with such properties. This is reminiscent of other methods in combinatorics, such as the Probabilistic Method or the Topological Method (See \cite{ALON2003} for a survey of these methods). 

\section{Notation}
We introduce some convenient notation here. We write that $A \lesssim_n B$ to mean that there exists some constant
$C(n)$ which depends on $n$ such that $A \leq C(n) B$. Further, we write that $A \sim_n B$ if $A \lesssim_n B$ and $B \lesssim_n A$.

We write $\text{Poly}_D (\KK^n)$ to represent the space of polynomials in $n$ variables with coefficients in a field $\KK$ and degree at most $D$.

The indicator function $\OO$ is defined on logical statements $X$ as follows:
\[
    \OO[X] = 
  \begin{cases}
      1 & \text{if X is true}, \\
      0 & \text{if X is false}.
  \end{cases}  
\]
For any function $f : \RR^n \to \RR$ let us denote the zero set of $f$ by $Z(f) = \{x \in \RR^n \ | \ f(x) = 0\}.$
We borrow from Computer Science the big O notation. For functions $f,g : \NN^+ \to \RR$ we write:
\begin{align*}
    f(N) &= O(g(N)) \iff \exists N_0, M \in \NN \text{ such that } f(n) \leq Mg(n) \forall n > N_0 \\
    f(N) &= \Omega(g(n)) \iff g(N) = O(f(N)).
\end{align*}
These can be thought of as asymptotic upper and lower bounds respectively.
\chapter{The Kakeya Problem in Finite Fields \label{chap:kakeya}}
Before we can discuss the Kakeya problem in finite fields, and its rather surprising resolution, we ought to first discuss the origin and history of the problem. 
Work on the Kakeya problem can be traced back to the Russian mathematician Abram Besicovitch in 1917. While working on a problem in Riemann integration, Besicovitch reduced it to the question
of the existence of planar sets of measure zero which contain a line segment in every direction. In 1920, Besicovitch constructed such a set and published in a Russian Journal.

However, 1917 was a turbulent year as it marked the end of
the Russian Empire and the start of the Russian civil war. Due to this and the ensuing blockade of Russian ports there was scarce communication with the outside world.
Thus Besicovitch could not have known of a Japanese mathematician Kakeya who asked also in 1917 a related question: What is the smallest area of a convex set within which
one can rotate a needle by 180 degrees in the plane? Julius Pal answered this question in 1921 with the equilateral triangle. The 
more interesting problem without the convexity condition remained open. In 1924, after leaving the newly formed Soviet Union for Copenhagen, Besicovitch discovered this
problem and by modifying his previous construction produced a solution in 1925. This lead to the more general questions being asked about Kakeya sets in higher dimensions.
\begin{definition}[Kakeya Set in $\RR^n$]
    A Kakeya set is a set $A \subset \RR^n$ that contains a unit segment in every direction.
\end{definition}
Besicovitch's construction showed that these sets can have arbitrarily small measures, even attaining zero, in $\RR^2$. Further, a straightforward construction produces these measure zero sets in dimensions $> 2$.

The natural question then arises, what is the dimension of such sets? There are many notions of dimensions that can be investigated, but we restrict ourselves to the Minkowski and Hausdorff dimensions.

\begin{definition}[Minkowski Dimension]
Given a set $S \subset \RR^n$, define $N(\varepsilon)$ to be the number of boxes of side length $\varepsilon$ required to cover the set.
The Minkowski Dimension of the set $S$ is then defined as:
$$\dim_M (S) = \lim_{\varepsilon \to 0} \frac{\log( N(\varepsilon))}{\log (1/\varepsilon)}.$$
If this limit does not exist, one can still define the upper and lower Minkowski dimensions, 
$\dim_{M_{\text{upper}}}$ and $\dim_{M_{\text{lower}}}$, by taking the limit superior and limit inferior respectively.

\end{definition}

\begin{definition}[Hausdorff Dimension]
    We define the $d$-dimensional Hausdorff measure of a set $S \subset \RR^n$ as: 
    $$\mathcal{H}^d(S)=\lim_{r \to 0} \inf \left\{\sum_i r_i^d:\text{ there is a countable cover of } S\text{ by balls with radii } 0 < r_i < r\right\}$$
    Then we can define the Hausdorff dimension of the set $S$ to be:
    $$\dim_H (S) = \inf \{ d \geq 0 : \mathcal{H}^d(S) = 0 \}.   $$
\end{definition}
These dimensions are related by the following inequality when they are all defined:
$$\dim_H \leq \dim_{M_{\text{lower}}} \leq \dim_{M_{\text{upper}}}.$$
In 1971, Davies produced a solution for the 2 dimensional case, proving that
although the measure of a Kakeya set can be arbitrarily small, it must have Hausdorff (and hence Minkowski) dimension of 2.\cite{davies1971some}
This resulted in the following conjectures:
\begin{conjecture}[Kakeya Conjecture for the Minkowski Dimension]
    Let $A$ be a Kakeya set in $\RR^n$. Then $\dim_M (A) = n$.
\end{conjecture}

\begin{conjecture}[Kakeya Conjecture for the Hausdorff Dimension]
    Let $A$ be a Kakeya set in $\RR^n$. Then $\dim_H (A) = n$.
\end{conjecture}
\todo{Wolff finite fields}


\section{Background}
Analogous to the Euclidean case, we define lines in $\FP^n$ as the set:
$$\mathcal{L} = \{x+ty : x,y \in \FP^n,  t \in \FP \}$$

A Kakeya set in $\FP^n$ is a set that contains a line in every direction.
\begin{figure}[h]
\centering 
\includegraphics[width=0.3\textwidth]{images/kakeya_ex_f32.png}}
\caption{An example of a Kakeya set (shaded) in $\FF_3^2$.}
{\label{kak_ex_f32}
\end{figure}
\todo{big example}


\section{Introduction to Finite Fields}
\begin{definition}[Finite Field]
    A finite field, $\FF$, is a finite set that forms a field. That is, it is closed under addition, subtraction, multiplication, and non-zero division. 
    The number of elements of a finite field, $|\FF|$, is called the order of the finite field.
\end{definition}
A finite field of order $q$ exists if and only if $q = p^k$ for some prime $p$ and integer $k$. 

\begin{lemma}
    Each element $X$ in a finite field $\FF$ satisfies the identity:
    $$X^{|\FF|} - X =0$$
    identically in $\FF$.
    \label{lem:finite-fields-poly-identity}
\end{lemma}
This lemma follows immediately from Fermat's Little Theorem.
\todo{more needed here}




\section{Combinatorial attempts at the proof} \todo{this section sucks, rewrite!}
We fix a finite field $\FF = \FF_{p^k}$.
\subsection{Bush Argument}
Bourgain produced one of the first non-trivial estimates of the dimension in 1991.\cite{BUSH1991} We present the finite field analogue of his argument here.\cite{GUTH2016}
\begin{theorem}[Bush Argument]
If $l_1, \dots, l_M$ are lines in $\FF^n$, then the number of points in their union is at least 
$$\frac12 M^{1/2} |\FF|$$
In particular, if $A$ is a Kakeya set, then we have:
$$|A| \gtrsim |\FF|^{\frac{n+1}{2}}$$
\end{theorem}

\prf{Let $X$ be the union of the lines $l_1, \dots, l_M.$ Each of these lines contains $|\FF|$ points of $X$,
so we have $|\FF|M$ points to distribute over $X$. By the pigeonhole principle, there is a point $x \in X$ which lies in at least $|\FF|M |X|^{-1}$ of the lines $l_i$.
\todo{ Im not sure what to expand on here - we could contradict by saying if this wasn't the case but that seems verbose}

These set of lines $l_i$ through $x$ is called the bush of $x$. These lines are disjoint except at $x$, and their union 
lies in $X$. So we have:
\[
(|\FF| -1) |\FF| M |X|^{-1} \leq |X|.
\]
Rearranging we get:
\[
\frac12 |\FF| M^{1/2} \leq |X|    
\]
A Kakeya set $A \subset \FF^n$ contains at least $ |\FF|^{n-1}$ lines. Setting $M= |\FF|^{n-1}$ yields:
\[
\frac12 |\FF||\FF|^{\frac{n-1}{2}} \sim |\FF|^{\frac{n-1 +2}{2}} =  |\FF|^{\frac{n+1}{2}} \lesssim |A|.
\]
}


\subsection{Hair Brush Argument}
Due to Wolff. \cite{WOLFF1995}

\begin{theorem}[Hair Brush Argument]
    Suppose $l_1, \dots , l_M$ are lines in $\FF^n$, and that at most $|\FF| +1$ of the lines lie in any plane.
    Then their union has cardinality at least 
    \[
        \frac{1}{3}    |\FF|^{3/2} M^{1/2}.
    \]
    In particular, if $A$ is a Kakeya set, then we have:
    \[
        |A| \gtrsim |\FF|^{\frac{n+2}{2}}    
    \]
\end{theorem}
\begin{proof}
    Let $X = \cup_i l_i$. If $l_i$ is a line in $A$, then the hairbrush with stem $l_i$ is
    defined to be the set of lines $l_j$ which intersect $l_i$. An average point of $X$ lies in $|\FF|M |X|^{-1}$ lines $l_i$.
    If each point of X was about average, then each hairbrush would contain $\gtrsim |\FF|^2 M |X|^{-1}$ lines. We claim
    that there is always at least one hairbrush with $\geq (1/2)|\FF|^2M |X|^{-1}$ lines.  

   \todo{ Finish this proof}
\end{proof}

\section{Dvir's Proof \label{sect:Dvirs-proof}}

In finite fields Kakeya's conjecture is as follows:
\thm{Kakeya Conjecture in Finite Fields}{If $A\subset \FP^n$ contains a translate of every line, then $|A| \gtrsim_n p^n $. \label{KakeyaConjecture}}

We shall prove this theorem via 3 surprisingly simple lemmas. This formulation of Dvir's proof is due to Gowers.\cite{GOW2020}

\begin{lemma}[Parameter Counting] 
Let $\KK$ be a (not necessarily finite) field. If $A \subset \KK^n$ and $|A| < {{n+D} \choose {n}}$, there exists a non-zero polynomial $P(x_1,\dots, x_n)$ of degree $D$ that vanishes on $A$. \label{lem:paramcounting}
\end{lemma}

\prf{ We first show the dimension of $\text{Poly}_D (\KK^n)$ is ${D+n} \choose{n}$. A basis for $\text{Poly}_D (\KK^n)$ is given by monomials of the form $x_1^{D_1}\dots x_n^{D_n}$, where $\sum_i D_i \leq D$, hence we just need to count the number of monomials.

We can map a monomial $x_1^{D_1}\dots x_n^{D_n}$ to a string of $D$ $\star$'s and $n$ $|$'s as follows. Begin with $D_1$ $\star$'s, then place one $|$. We put now $D_2$ $\star$'s, and place a second $|$. We continue until we have placed $D_n$ $\star$'s followed by an $n^{\text{th}}$ $|$. Finally we place $D - \sum_i D_i \star$'s. This is a bijective map between the monomials in $\text{Poly}_D (\KK^n)$ and all the strings of $D$ $\star$'s and $n$ $|$'s. To count the strings, fix the $n$ $|$'s. Now we have $n+1$ bins to distribute our $D$ $\star$'s. 
Therefore we have by the stars and bars theorem:
$$\text{Poly}_D (\KK^n) = {{n+1 + D - 1} \choose{n+1 -1}} = {{n+D} \choose{n}}.$$

Let now $p_1, \dots, p_{|A|}$ be the points of $A$. We consider the evaluation map $E: \text{Poly}_D (\KK^n) \to \KK^{|A|}$ defined by:
$$E(Q) = \left(Q(p_1), \dots, Q(p_{|A|})\right).$$

This map is clearly linear. Its kernel $\ker E$ is exactly the set of polynomials in $\text{Poly}_D (\KK^n)$ that vanish on $A$. By assumption, the dimension of $\text{Poly}_D (\KK^n)$ is greater than $A$, so the dimension of the domain of $E$ is greater than the codomain of $E$. By the rank-nullity theorem, we conclude $E$ must have a non-trivial kernel. Thus there exists a non-zero polynomial $P \in \text{Poly}_D (\KK^n)$ that vanishes on $A$.} 


Note that if $D=|\FF|-1$, and $|A| \leq {{|\FF|+n-1} \choose{|\FF|-1}} = { {|\FF|+n-1} \choose{n}}$ we have a polynomial of degree $|\FF|-1$ that vanishes on $A$. Since $\frac{|\FF|^n}{n!} < {{|\FF|+n-1} \choose{n}}$,we can definitely find such a polynomial when $|A| \leq \frac{|\FF|^n}{n!}$.


\begin{lemma}Suppose $A \subset \FF^n$ contains a line in every direction, and suppose that there exists a non-zero polynomial $P$ with degree $D<|\FF|$ that vanishes on $A$. Then there exists a non-zero degree $D$ polynomial $\bar{P}$ that vanishes everywhere on $\FF^n$. \label{lem:kaklem2} \end{lemma}

\prf{ Choose a line in $A$, say $\ell = \{x+tz : t\in \FF \}$ with $x\in \FF^n$ and $z \in \FF^n / \FF^{\times}$. Now we consider the restriction of our polynomial $P$ to the line $\ell$, $P_{|\ell}$ .
Recall $P$ is a sum of monomials, and we use multi-index notation here with $\alpha = (\alpha_1, \alpha_2, \dots, \alpha_n), \ \alpha_i \in \NN \cup \{0\}$ and $|\alpha| =  \sum \alpha_i$. $P$ can be written as:
$$P(x_1,x_2, \dots, x_n) = \sum_{|\alpha| \leq D} c_{\alpha} x_1^{\alpha_1}x_2^{\alpha_2}\dots x_n^{\alpha_n}. $$
Now $P_{|\ell}$ can be written:
$$P_{|\ell} = P(x+tz) = Q_{x,z}(t) = \sum_{|\alpha| \leq D} c_\alpha \prod_{i} (x_i + t z_i)^{\alpha_i}.$$
We now wish to examine the degree $D$ term of $Q$, which is achieved by picking the $tz_i$ terms from each bracket in the product above. This gives the degree $D$ component of $Q$, $Q_{x,z,D}$, which has the form:
$$Q_{x,z,D} = t^D Q_D(z) = t^D \sum_{|\alpha| = D} c_\alpha \prod_i z_i.$$

Now if $P_{|\ell}$ vanishes everywhere on $\ell$, since its dependence on $t$ is given by a polynomial of degree less than $|\FF|$, all its coefficients must be zero. 
This is clear from the factor theorem, as we could write the roots of $P_{|\ell}$ as $(t-k_1)(t-k_2)\dots(t-k_{|\FF|})$, but this contradicts the fact $P$ is of degree $D < |\FF|$.

Notice that $Q_{x,z,D}$ no longer depends on $x$, but on $z$ alone. In particular $Q_D(z) = 0$, but $z$ was an arbitrary element of $\FF^n / \FF^{\times}$, and $Q_D(z)$ also vanishes at zero, so it vanishes everywhere. Thus we can pick $\bar{P}$ = $Q_D$, and we are done.
}





\lem{Let $P$ be a non-zero polynomial on $\FF^n$ with degree less than $|\FF|$. Then $P$ does not vanish everywhere. \label{lem:kaklem3}}

\prf{ We proceed by induction on $n$. For $n=1$, a non-zero polynomial that vanishes everywhere has $|\FF|$ roots, so must be at least of degree $|\FF|$.
Let us assume that the statement holds in $\FF^{n-1}$, we now prove it must also hold for $\FF^N$.

We let $x_1, \dots ,x_n$ be coordinates on $\FF^n$, and we write $P$ in the form:

$$P(x_1, \dots ,x_n) = \sum_{j=n}^{|\FF|-1} P_j(x_1, \dots x_{n-1}) x_n^j.$$

Each $P_j$ are polynomials in $x_1, \dots x_{n-1}$ of degree less than $|\FF|$. Fix $x_1, \dots x_{n-1}$, and let $x_n$ vary. Now we have a polynomial in $x_n$ of degree less than $|\FF|$ that vanishes for all $x_n \in \FF$. By the base case this must be the zero polynomial. So each $P_j(x_1, \dots, x_{n-1}) = 0$ for all $j$ and for all $(x_1, \dots x_{n-1}) \in \FF^{n-1}$. Now by induction on $n$, each $P_j$ is the zero polynomial. Then $P$ is the zero polynomial as well. 

}

\begin{proof} [Proof of Theorem \ref{KakeyaConjecture}] 
Assume $A \subset \FF^n$ is a Kakeya set, and that $|A| \leq \frac{|\FF|^n}{n!}$. Then by \ref{lem:paramcounting} we can find a 
non-zero polynomial, say $P$, that vanishes on $A$. Now by \ref{lem:kaklem2} there exists a non-zero polynomial $\bar{P}$ that vanishes everywhere on $\FF^n$, and has degree less than $|\FF|$.
Finally \ref{lem:kaklem3} says that such a $\bar{P}$ is necessarily the zero polynomial, a contradiction. We conclude that $|A| > \frac{|\FF|^n}{n!}$, or in other words:
$$
|A| \gtrsim_n |\FF|^n.
$$
\end{proof}
\chapter{The Polynomial Method in Additive Combinatorics \label{chap:alon}}
So far our discussion has only centred around the usefulness of polynomial methods in geometric combinatorial problems. 
In this chapter we shall discuss applications to the exciting field of additive combinatorics, where in 1990 Alon provided perhaps
the first example of the polynomial method in action.\cite{alon1999combinatorial} We present Michalek's short, elementary, and direct proof of the combinatorial nullstellensatz.\cite{michalek2010}
\section{Combinatorial Nullstellensatz}
\begin{theorem}[Combinatorial Nullstellensatz]
    Let $\KK$ be a (not necessarily finite) field, and let $P(X_1,\dots, X_n) \in \KK[X_1,\dots, X_n]$ be a polynomial in $n$ variables with coefficients in $\KK$. Suppose the coefficient of $x_1^{k_1}x_2^{k_2}\dots x_n^{k_n}$ is non-zero and further suppose $\deg P = \sum_{i=1}^n k_i$, where each $k_i$ is a non-negative integer.

    Then for any subsets $A_1,\dots A_n$ of $\KK$ satisfying $|A_i| > k_i$ for each $1\leq i\leq n$ there exist $a_1 \in A_1, \dots, a_n \in A_n$ such that $P(a_1,\dots, a_n) \neq 0$.
    \label{comb-nullstellensatz}
\end{theorem}
\begin{proof}
    We proceed by induction on $\deg P = D$. When $D=1$, $P$ is simply a linear combination of $n$ variables say $P(X_1,\dots X_m) = c_1 X_1 + \dots c_n X_n$. Without loss of generality assume $x_1$ has a non-zero coefficient and consider the sets $A_i = \{a_{i,1}, a_{i,2}\}$. Suppose $P$ at the point $(a_{1,1}, a_{2,1}, \dots, a_{n,1})$ is zero. We can then determine $c_{1} =- \frac{c_2 a_{2,1} + c_3 a_{3,1}+\dots+ c_n a_{n,1}}{a_{1,1}}$. Now evaluating at $(a_{1,2}, a_{2,1}, \dots, a_{n,1})$ we see that this is zero only when $a_{1,1} = a_{1,2}$, so our theorem holds. 

    Now let us assume the theorem holds for $\deg P = D -1$, and prove for $\deg P = D$.
    Suppose that $P$ satisfies the assumptions of the theorem but $P(x) = 0$ for every $x \in A_1 \times \dots \times A_n$.
    Without loss of generality $k_1 > 0.$ Fixing $a\in A_1$ we can write
    \[
    P = (x_1-a)Q +R  \tag{$\dagger$}
    \]
    by the usual long division of polynomials. The degree of $R$ in $x_1$ must be strictly less than $\deg(x_1-a)$, so $R$ does not contain any
    $x_1$ terms. Thus it follows that $Q$ must have a monomial with non-zero coefficient of the form $x_1^{k_{1} -1} x_2^{k_2} \dots x_n^{k_n}$ and 
    $\deg (Q) = D-1$.

    Take any $x \in \{a\} \times A_2 \times \dots \times A_n$ and evaluate $(\dagger)$. Since $P(x) = 0$ it follows that $R(x) = 0$, but $R$ is independent of $x_1$ so $R$ must also vanish on $A_1 \backslash \{a\} \times A_2 \times \dots \times A_n$.
    Now take any $x \in A_1 \backslash \{a\} \times A_2 \times \dots \times A_n$ and evaluate $(\dagger)$. Since $(x_1 - a)$ is non-zero, $Q(x) =0$. So $Q$ vanishes on all $x \in A_1 \backslash \{a\} \times A_2 \times \dots \times A_n$, which contradicts the inductive hypothesis.
\end{proof}
\section{Cauchy-Davenport Theorem}
\begin{theorem} [Cauchy-Davenport Theorem]
    Let $A,B$ be non-empty subsets of $\ZZ_p$ for some $p$ prime. Define their sumset $A+B$ as follows:
    \[
    A + B = 
    \left\{ x \in \ZZ_p\ | \ x = a+b \text{  for some } a\in A, \ b \in B \right\}.
    \]
    Then we have:
    \[
    |A+B| \geq  \min \left\{p, |A| + |B| -1 \right\}.
    \]
\end{theorem} \todo{Generalise this!}

\begin{proof}
    Let us tackle the two cases separately. First, assume that $\min \left\{p, |A| + |B| -1 \right\} = p$.
    Then if $|A| + |B| > p$, $A$ and $B$ must intersect. \todo{ More explanation on $\cap$ needed? PHP?}
    For some $g \in \ZZ_p$ denote the set  $\left\{ g - x  \ | \ x \in B, \  \right\} \subset \ZZ_p$ as $g-B$. Since $|g-B| = |B|$, we have that 
    $g-B$ and $A$ must intersect as well. Thus there exists some $a \in A$, $b \in B$ such that: 
    \begin{align*}
        g& -b = a \\
        g& = a+b.
    \end{align*}
    Our choice of $g$ was arbitrary, so it follows that $A+B = \ZZ_p$ and hence $|A+B| = p$.

    Now assume that $\min \left\{p, |A| + |B| -1 \right\} = |A| + |B| -1$. Then if the theorem is false we have $|A+B| \leq |A| + |B| -2$, so there exists some $C \subset \ZZ_p$ such that 
    $A+B \subset C$ and $|C| =  |A| + |B| -2$. Now let us define a polynomial $f(x,y) \in \ZZ_p [x,y]$ as:
    \[
        f(x,y) = \prod_{c \in C} (x+y -c).
    \]
    Since $A+B \subset C$, $f(a,b) =0$ for all $(a,b) \in A\times B$. Further, the degree of $f$ is $\deg f = |C| = |A| + |B| -2$. 
    We can now appeal to the combinatorial nullstellensatz to yield a contradiction. Let $k_1 = |A| -1$, and $k_2 = |B|-1$. 
    Now $\deg f = k_1 + k_2$, and the coefficient of $x^{k_1}y^{k_2}$ is
    ${|A|+|B| - 2} \choose {|A|- 1}$ which is non-zero in $\ZZ_p$ as the numerator cannot contain a factor of $p$ by assumption. Applying Theorem \ref{comb-nullstellensatz} we see that there 
    must exist some $(a,b) \in A \times B$ such that $f(a,b) \neq 0$, a contradiction.
\end{proof}
\chapter{The Joints Problem}

\section{Background}
Let $\LL$ be a set of distinct lines in $\RR^n$. A joint of $\LL$ is a point which lies in three non-coplanar lines of $\LL$.
The joints problem consists of setting a sharp upper bound on the maximal number of joints that can be formed from a configuration of $L$ distinct lines.
We denote this quantity $J(L)$. 

We shall begin by examining an example based on a grid, with the hopes of gaining better intuition about the problem and formulating a conjecture. 
\begin{example}Consider an $N \times N \times N$ regular grid of integer coordinates. We shall give a collection of lines such that each point of this grid is a joint for the collection.
Let $\LL$ be the collection of all lines parallel to any of the Cartesian axes that intersect this a point in this grid.
For each horizontal $N \times N$ layer, there are $N+N = 2N$ lines that intersect our grid. 
There are $N$ layers, so we obtain $2N^2$ distinct lines in this manner. Finally we need to account for the $N^2$ lines perpendicular to the $N\times N$ layers.
This leaves us with $|\LL| = 3 N^2$ lines forming $N^3$ joints. The number of joints is thus $\sim$ $|\LL|^{3/2}$. 
\end{example}
\begin{figure}[h]
    \centering
\begin{tikzpicture}
    \draw [step=0.5] (0.25,0.25) grid (5.25,4.25);
\end{tikzpicture}
\caption{A $N \times N$ layer of our grid.}
\end{figure}

We can extend this example to higher dimensional grids easily. 
\begin{example}
    

If we have an $\underbrace{N \times \dots \times N}_{n \text{ Dimensions}}$ regular grid of integer coordinates in $\RR^n$, we can construct an example  by a straightforward extension
of the above example. Each additional dimension increases the number of lines by a factor of $N$, this can be seen by considering each new dimension as a layering of the previous set along the new axis.
Thus we can see that $\sim N^{n-1}$ lines form $N^n$ joints in this manner. So the number of joints is $\sim |\LL|^{\frac{n}{n-1}}$.
\end{example}

It turns out that the examples illustrated above provide asymptotically maximal configurations, that is, disregarding the best constant $C$ such that $J(L) \leq CL^{\frac{n}{n-1}}$.
\section{Solution of the Joints Problem}
This solution was first produced by Guth-Katz for the three-dimensional case,\cite{guth2008algebraic} and later extended to the general case by Quilodrán,\cite{quilodran2009joints} and independently at the same time by 
Kaplan-Sharir-Shustin.\cite{kaplan2009lines}


\begin{theorem}
    Any $L$ lines in $\RR^n$ determine $\lesssim_n L^{\frac{n}{n-1}}$ joints.
\end{theorem}
We begin with the fundamental lemma to this proof. 
\begin{lemma}
    If $\LL$ is a set of lines in $\RR^n$ that determines $J$ joints, then one of the lines contains at most $nJ^{\frac{1}{n}}$ joints.
\label{joints_bound}
\end{lemma}
\prf{
Let $P$ denote the lowest degree non-zero polynomial that vanishes at every joint of $\LL$. By parameter counting, Lemma \ref{paramcounting}, the degree of $P$ is at most 
$nJ^{\frac{1}{n}}$. (To see this, set $D = \lfloor nJ^{\frac{1}{n}}\rfloor$ and then $J < {{D+n}\choose {n}}$.)

We proceed by contradiction. Assume every line contains more than $nJ^{\frac{1}{n}}$ joints.
 Now $P$ must vanish on every line in $\LL$ as the degree of $P$ is less than the number of joints it must interpolate.

We now examine the gradient of $P$ at each joint in $\LL$. We will need a fact about gradients for this, which we will encapsulate in the following lemma for clarity.
\begin{lemma}
    If $x$ is a joint of $\LL$, and if a smooth function $F: \RR^n \to \RR$ vanishes on the lines of $\LL$, then $\nabla F$ vanishes at $x$. 
\end{lemma}
\begin{proof}
    The joint $x$ is contained in $n$ non-coplanar lines $l_1, \dots, l_n$, in directions $v_1, v_2, \dots , v_n$ respectively. 
    Now consider the directional derivative for a particular $v_i$:
    \[
    \frac{\partial F}{\partial v_i} = \lim_{t \to 0} \frac{ \overbrace{F(x+ tv_i)}^{F \equiv 0 \text{ on a line in $\LL$}} - \overbrace{F(x)}^{F \equiv 0 \text{ on joints}}}{t} = \frac{0}{t} = 0.    
    \]
    Notice that $ \frac{\partial F}{\partial v_i} = \langle \nabla F, v_i\rangle$, so since we have this for each $v_i$, and the set of $v_i$'s form a basis of $\RR^n$, we have that $\nabla F(x) = 0$.
\end{proof}
So we see that the partial derivatives of $P$ vanish at each joint. The derivatives are polynomials of
smaller degree than $P$ and since $P$ was assumed to be the minimal degree non-zero polynomial that
vanishes at each joint, each derivative of $P$ is identically zero. This implies $P$ must be
constant, which implies that there does not exist such a minimal degree polynomial,
a contradiction.
}
Finally we can prove the main result. 
\prf{
Lemma \ref{joints_bound} tells us that if we remove a line from our collection, we are removing at most 
$n J(L) ^{\frac{1}{n}}$ joints. By repeating this process, we get the chain of inequalities:

\begin{align*}
    J(L) &\leq J(L-1) + n (J(L))^{\frac{1}{n}}\\
    &\leq J(L-2) + 2\left[ n (J(L))^{\frac{1}{n}}\right]\\
    &\leq J(L-3) + 3\left[ n (J(L))^{\frac{1}{n}}\right]\\
&\vdots\\
    &\leq L \left[ n (J(L))^{\frac{1}{n}}\right].
\end{align*}
Now we have:

\eq{   J(L) \leq L \left[ n (J(L))^{\frac{1}{n}}\right] }
   \eq{J(L)^ {\frac{n-1}{n}} \lesssim_n L}
   \eq{J(L) \lesssim_n L ^{\frac{n}{n-1}}}
}

\chapter{Szemerédi–Trotter Theorem}

In this chapter we will study the application of the polynomial method to incidence geometry by proving a fundamental theorem in the field.
Incidence geometry is the study of possible intersection patterns of simple geometric objects, such as lines or low degree curves. 
We have already seen an incidence problem in the previous chapter on the Joints problem. 
By developing the powerful tool of polynomial partitioning we shall see the key role that the topology of $\RR$ plays in such problems, in contrast to the trivial topology of finite fields. 


\section{Background}



\begin{theorem}[Szemerédi–Trotter]
    Let $\PP \subset\RR^2$ be a finite set of points and
    let $\LL \subset \RR^2$ be a finite set of lines. We define 
    \[I(\PP,\LL) = \{(p, \ell) \in \PP \times \LL \ | \ p \in \ell\}\] 
    to be the set of incidences between $\PP$ and $\LL$. 
   
    Then:
    \[
        |I(\PP, \LL)| \lesssim \left(|\PP||\LL|\right)^{2/3} + |\PP| + |\LL|
    \]
    
    \label{thm:S-T}
\end{theorem}

\section{The Trivial Bound}
In planar geometry, we have the following dual statements: two points determine a line and every pair of lines intersect in at most one point.
Using this we can prove the following bounds on $I(\PP,\LL)$:
\begin{theorem}[Trivial Incidence Bounds]
    $$I(\PP,\LL) \lesssim |\PP|\cdot |\LL|^{\frac{1}{2}} + |\LL|$$
    and
    $$I(\PP,\LL) \lesssim |\LL|\cdot |\PP|^{\frac{1}{2}} + |\PP|.$$
    \label{thm:trivial-ST-bounds}
\end{theorem}
\begin{proof}
    

We have that 
$$|I(\PP,\LL)| \leq |\PP|^2 + |\LL|.$$
To see this, count the lines that have at most one point in $P$ on them. These contribute at most $|\LL|$ incidences.
Every other line has at least two points in $\PP$. The total number of incidences on these lines is at most $|\PP|^2$
as otherwise there would exist a $p\in \PP$ that lies on over $|\PP|$ lines, and each of these lines would have an additional 
point on it. This would imply there are more that $|\PP|$ points, a contradiction. 

We now bound the number of incidences. 

\begin{align*}
    |I(\PP,\LL)|^2 &= \left( \sum_{\ell \in \LL} \sum_{p \in \PP} 1_{p\in \ell} \right)^2 \\
    & \leq |\LL|\cdot \sum_{\ell \in \LL} \left( \sum_{p \in \PP} 1_{p\in \ell} \right)^2  \qquad \text{ (Cauchy-Schwarz on $\ell$)}\\
    &= |\LL|\cdot \sum_{p_1,p_2 \in \PP} \sum_{\ell \in \LL}   1_{p_1\in \ell} 1_{p_2\in \ell}    \\
    & \leq |\LL|\cdot ( |I(\PP,\LL)| + |\PP|^2)\\
    &\leq |\LL|^2 + 2|\LL|\cdot |\PP|^2 \qquad \text{ (Applying trivial bounds \ref{thm:trivial-ST-bounds})}\\ 
\end{align*}    
This implies
$$I(\PP,\LL) \lesssim |\PP|\cdot |\LL|^{\frac{1}{2}} + |L|.$$ 
Repeating the above proof interchanging the roles $\PP$ and $\LL$ achieves the other bound.
\end{proof}


\section{Examples}
We can not improve beyond trivial bounds in a finite field $\FF^2$. Unlike the joints problem, where the results coincided for finite fields and the reals,
this suggests that the topology of $\RR$ plays a special role. 
\begin{example}[Finite Fields]
Consider the set of points $\PP=\FF^2$ and lines 
$\LL = \FF^2$. Every line contains exactly $|\FF|$ many points in $\FF^2$, so we have $|\FF|^3$ incidences. 
So both sides of the trivial inequality \ref{thm:trivial-ST-bounds} are comparable:
$$
I(\PP,\LL) = |\FF|^3 \sim (|\FF|^2)(|\FF|^2)^{1/2} + |\FF|^2.
$$
\end{example}
In contrast, the following examples seem to be the best possible over $\RR$. We will later prove that these are the tight case of Theorem \ref{thm:S-T}.
\begin{example} Consider the following collections in $\RR^2$:
    \begin{align*}
    \PP &= \{ (a,b) \in \ZZ^2 \ : \ a\in [1,N], b \in [1,2N^2] \} \\
    \LL &= \{ (x,mx+c) \in \RR^2 \ : \ m,c\in \ZZ, m\in [1,N], c \in [1,N^2] \} \end{align*}
    We have $P= 2N^3$ points and $L = N^3$ lines. Every line in $\LL$ contains $N$ points in $\PP$ so there are
    $N^4$ incidences. Both sides Szemerédi-Trotter are comparable as 
    $$ I(\PP,\LL) = N^4 \sim (N^3)^{\frac{2}{3}} (N^3)^{\frac{2}{3}} \sim P^{2/3} L^{2/3}$$
\end{example}
\todo{diagram?}
\begin{example}
    $R$ \& $2R$ example
\end{example}
\todo{add this example}



\section{Ham Sandwich Theorems}
\begin{theorem}[General Ham Sandwich Theorem]
Let V be a vector space of continuous functions on $\RR^n$. Let $U_1,U_2,\dots,U_N \subset \RR^n$ be finite volume open sets with $N< \dim V$. For any function $f \in V\backslash \{0\}$, suppose $Z(f)$ has Lebesgue measure zero.

Then there exists a function $f \in V\backslash \{0\}$ that bisects each $U_i$. \label{thm:GenHamSand}
\end{theorem}
\begin{proof}
Define the functions $\{\phi_i\}_{i=1}^N$, $\phi_i: V\backslash \{0\} \to \RR$ by
\[
\phi_i(F) = \text{Vol}(\{x\in U_i | F(x) > 0 \}) - \text{Vol}(\{x\in U_i | F(x) > 0 \})
\]
We notice that each $\phi_i(F) = 0$ if and only if $F$ bisects $U_i$. Notice also that $\phi_i(-F) = -\phi_i(F)$, hence $\phi_i$ is antipodal. We now show each $\phi_i(F)$ is continuous.
It is enough to show that is $U$ is a finite volume open set, that the measure of $\{x\in U| f(x)>0\}$ depends continuously on $f\in V \backslash \{0\}$.

Suppose $f_n \to f$ in $V$ for some $f,f_n \in V \backslash \{0\}$. $f_n$ converges to $f$ in the topology of $V$, so it follows it must converge pointwise. Pick any $\varepsilon >0$. We can find a subset $E\subset U$ so that $f_n \to f$ uniformly pointwise on $U \backslash E$, and $m(E)< \varepsilon$.
By hypothesis, $m(Z(f)) =0$ and $m(U) < \infty$. Hence we can choose $\delta$ such that $m\left(\{x\in U| |f(x)|<\delta\}\right) < \varepsilon$.

Now we choose $n$ sufficiently large such that $|f_n (x) - f(x)| < \delta$ on $U\backslash E$. Then we have \[|m\left(\{x\in U| f_n(x)>0\}\right) - m\left(\{x\in U| f(x)>0\}\right)| < 2 \varepsilon.\] Since $\varepsilon$ was arbitrary each $\phi_i$ is continuous.

We now combine each $\phi_i$ into the map $\phi : V\backslash \{0\} \to \RR^N$. Since $\dim V > N$, let $\dim V = N+1$. Now choose an isomorphism of $V$ with $\RR^{N+1}$, and think of $S^N$ as a subset of $V$.
Now the map $\phi: S^N \to \RR^N$ is antipodal and continuous. By the Borsuk-Ulam theorem, there exists an $F\in S^N \subset V\backslash \{0\}$ such that $\phi(F) = 0$.
\end{proof}

\begin{corollary}[Finite Ham Sandwich Theorem]
    Let $S_1, \dots , S_N$ be finite sets in $\RR^n$ with $N < {{D+n}\choose{n}} = \text{Poly}_D (\RR^n)$. Then there exists a non-zero $P\in \text{Poly}_D (\RR^n)$ that bisects each $S_i$.  \label{thm:FiniteHamSandwich} 
\end{corollary}

\begin{proof}
    For each $\delta>0$, define $U_{i, \delta}$ to be the union of $\delta-balls$ centred at the points of $S_i$. By Theorem \ref{thm:GenHamSand}, we can find a non-zero $P_{\delta}$ with degree less than $D$ that bisects each $U_{i, \delta}$. By rescaling we can assume $P_{\delta} \in S^N \subset \text{Poly}_D (\RR^n) \backslash \{0\}$.
    Since $S^N$ compact, we can find a sequence $\delta_m \to 0$ so that $P_{\delta_{m}}$ converges to $P$ in $S^N$. Since the coefficients of $P_{\delta_{m}}$ converge to $P$, $P_{\delta_{m}}$ converges to $P$ uniformly on compact sets.

    We claim $P$ bisects each $S_i$. By contradiction, suppose $P>0$ on more than half the points of $S_i$, say on the points of $S_i^+$. By choosing $\varepsilon$ sufficiently small, we can assume $P>\varepsilon$ on the $\varepsilon$-ball around each point of $S_i^+$.
    Further, we can choose $\varepsilon$ such that each $\varepsilon$-ball is disjoint. 

    Since $P_{\delta_{m}}$ converges uniformly, we can find $m$ sufficiently large such that $P_{\delta_{m}}>0$ on the $\varepsilon$-ball around each point of $S_i^+$. By making $m$ large, we can also arrange that $\delta_n < \varepsilon$.
    Thus $P_{\delta_{m}} > 0$ on more than half the points of $U_{i, \delta_{m}}$.
\end{proof}
\section{Proof of Szemerédi–Trotter Theorem}
\begin{theorem}[Polynomial Partitioning]
   For any $n$ there exists a constant $c(n)$ such that if $S$ is a finite subset of $\RR^n$ and $D$ is any degree, then there exists
   a polynomial $P$ of degree $D$ such that $\RR \backslash Z(P)$ is a disjoint union of $\lesssim D^n$ open sets $O_i$ each containing
   $\leq c(n) |S|D^{-n}$ points. \label{thm:PolyPartioning}
\end{theorem}

\begin{proof}
We repeatedly apply Corollary \ref{thm:FiniteHamSandwich}. We begin by finding a polynomial $P_1$ of degree 1 that bisects $S$. We partition $\RR \backslash Z(P_1)$
into two disjoint open sets according to the sign of $P_1$, each containing at most $|S|/2$ points. We then bisect both of these sets using another polynomial $P_2$. There are four sign conditions on $P_1$ and $P_2$, and the subset for
each sign condition contains at most $|S|/4$ points of $S$. Continuing this process to define polynomials $P_3, P_4, \dots$, the polynomial $P_j$ bisects $2^{j-1}$ finite sets. By Corollary \ref{thm:FiniteHamSandwich}, we can find $P_j$ with degree
$\lesssim 2^{j/n}$. $\RR \backslash Z (P_1 \cdot P_2 \cdot \dots P_j)$ is the disjoint union of $2^j$ open sets each containing $\leq |S| 2^{-j}$ points. 
Repeating this procedure $J$ times, and defining $P = \prod_{i=1}^{J} P_i$, $\RR^n \backslash Z(P)$ is the disjoint union of $2^J$ open sets each containing $ \leq |S|2^{-J}$
points of $S$. Now we choose $D$ such that $\deg(P) < D$ which is equivalent to $\sum_{j=0}^J c(n) 2^{j/n} \leq D$. But $\sum_{j=0}^J 2^{j/n}$ is a geometric series so we can find $\deg (P) < D$ for $D \leq c(n) 2^{J/n}$. 
The number of points in each $O_i$ is $\leq |S| 2^{-J} \leq c(n) |S| D^{-n}$
\end{proof}

\begin{lemma} [Trivial Bounds]
    By double counting, we have the following trivial estimates:
    $$
    I(\ES, \LL) \leq L^2 + S,
    $$
    $$
    I(\ES, \LL) \leq S^2 + L.
    $$
    
    
\end{lemma}
\begin{proof}
We need only consider the case $S^{\frac{1}{2}} \leq L \leq S^2$, as otherwise the proof follows immediately from the lemma above. Let $D$ be a degree to be chosen later. By Theorem \ref{thm:PolyPartioning}, there
exists a polynomial $P$ of degree $D$ such that each component of $\RR^2 \backslash Z(P)$ has $\lesssim SD^{-2}$ points. Let $O_i$ denote these components and $\ES_i = \ES \cap O_i$, $\LL_i = \LL \cap O_i$. 
Note that $\ES = \ES_c \cup \ES_z$, $\LL = \LL_c \cup \LL_z$, where $\ES_z, \LL_z$ are the set of points and lines in $Z(P)$ respectively.
$$I(\ES, \LL) \leq I(\ES_c, \LL) + I(\ES_z, \LL_z) + I(\ES_z, \LL_c)$$
If a line $\ell \notin Z(P)$ then it can intersect $P$ at most $D$ times, and so each line intersects at most $D+1$ cells. Hence $\sum L_i \leq (D+1)L$.
\begin{align*}
    I(\ES_c, \LL) &= \sum_i I(\ES_i, \LL_i) \leq \sum_i \ES_i^2 + \sum_i \LL_i\\
    &\lesssim LD + SD^{-2} \sum_i S_i \leq LD + S^2D^{-2}
\end{align*} 
We also have by our lemma
$$I(\ES_z, \LL_z) \leq S + D^2$$
and finally $$I(\ES_z, \LL_c) \leq LD.$$
Together we have now 
$$I(\ES, \LL) \lesssim LD + S^2D^{-2} +S + D^2.$$
We optimise $D$ in $LD + S^2D^{-2}$ by making both terms comparable and hence $D \sim S^{\frac{2}{3}} L^{-\frac{1}{3}}$. From our restriction $S^{\frac{1}{2}} \leq L \leq S^2$ we have $S^{\frac{2}{3}} L^{-\frac{1}{3}} \geq 1$
and $D^2 \sim S^{\frac{4}{3}} L^{-\frac{2}{3}} \leq S$ so we achieve
$$I(\ES, \LL) \lesssim (SL)^{2/3} + S $$
\end{proof}
\chapter{The Circle Tangency Counting Problem}
\todo{Kakeya proof with a twist}
Here we shall discuss the problem of counting the number of tangencies in a suitably non-degenerate collection of circles. We say two circles are tangent if their intersection contains a single point.\todo{is this stupid?}
The set of unordered pairs of circles in a collection $\CE$ which are mutually tangent are called the tangencies of the collection and is denoted $\tau(\CE)$. 
\section{Trivial Bounds}
\begin{theorem}[Trivial Bound]
    Let $\CE$ be an arbitrary finite collection of circles. Then the number of tangencies $|\tau(\CE)|$ is bounded as follows:
    \[
        |\tau(\CE)| \leq |\CE|^2
    \]
\end{theorem}
Stated for arbitrary collections of circles the problem is not particularly interesting as the above bound turns out to be asymotopically tight. 
\begin{example}
Denote a circle centred at $(x,y)$ with radius $r$ as $\gamma (x,y,r)$. \todo{notation}
Consider the following collection of $N$ circles:
\begin{align*}
    C_0 &= \gamma\left(0,0,1\right) \\
    C_1 &= \gamma\left(\frac{1}{2},0,\frac{1}{2}\right) \\
    C_2 &= \gamma\left(\frac{3}{4},0, \frac{1}{4}\right) \\
    &\vdots \\
    C_N &= \gamma\left(1- \frac{1}{2^N}, 0, \frac{1}{2^N}\right)
\end{align*}
Each circle in our collection $\CE = \cup_{0 \leq i \leq N} C_i$ is tangent to $N$ other circles at the point (1,0). 
Hence $|\tau(\CE)| \sim N^2 \sim |\CE|^2$. 
\todo{diagram}
\end{example}

Instead we shall look at collections of circles that satisfy a single non-degeneracy condition. We consider collections of circles 
such that no three are mutually tangent at a common point. The best known examples leverage our tight \hyperref[thm:S-T]{Szemerédi-Trotter Theorem} examples:
\begin{example}
    Let $\ES$, $\LL$ be collections of $N$ points and $N$ lines respectively such that the \hyperref[thm:S-T]{Szemerédi-Trotter Theorem}'s bound is tight. 
    In particular, they determine $\sim N^{4/3}$ point line incidences. 
    Let $C_1$ denote the collection of unit circles centred at the points of $\ES$. 
    Let $C_2$ denote the collection of lines obtained by translating each line $\ell in \LL$ one unit in the $\ell^{\perp}$ direction.
    If $(p,\ell) \in \ES \times \LL$ is a point-line incidence from our original collection, then the corresponding circle-line pair will be tangent.
    Performing an inversion transform about a point that does not lie in any of the circles or lines, our collection $\CE = C_1 \cup C_2$ becomes a 
    collection of $2N$ circles that determine $N^{4/3}$ tangencies.
 \end{example}

We shall now present a bound from a recent paper of Ellenberg-Solymosi-Zahl which uses the polynomial method.\cite{ellenberg2016new}

\begin{theorem}
    Given a finite collection $\CE$ of $N$ circles in the plane such that no three are tangent at a common point, 
    then the number of tangencies $|\tau(\CE)|$ obeys:
    \[
        |\tau(\CE)| \lesssim N^{3/2}
    \]
\end{theorem}

It is not obvious how to apply the polynomial method in the current formulation. 
We will perform a lifting  of our circles into algebraic curves $\RR^3$, 
preserving tangencies between circles in the plane as intersections between their lifted curves in $\RR^3$. 
This transforms the problem from a tangency problem to an incidence problem, and reduces the required degree of polynomial needed to interpolate.

We shall now discuss this lifting in detail. Let $\gamma \subset \RR^2$ be a circle in the plane of radius $r_{\gamma}$ centred at $(x_\gamma, y_\gamma)$. 
We define the lifting transform of the circle as: \[
    \beta(\gamma) = \left \{ (x,y,z) \in \RR^3 \ | \ (x-x_\gamma)^2 + (y-y_\gamma)^2 = r_\gamma^2, \ z = - \frac{x-x_\gamma}{y-y_\gamma} \right \}.
\]  
Clearly $\beta$ is an algebraic curve, so now we shall examine the correspondence between mutually tangent circles. Notice that $z$ is defined as the slope of the tangent line
at the point $(x,y) \in \gamma$.

\begin{lemma}
    Let $\beta$ be the transform defined as above. 
    Then two circles $\gamma, \gamma' \subset \RR^2$ are tangent if and only if $\beta(\gamma) \cap \beta(\gamma') \neq \emptyset$.
    \label{lem:beta-lift}
\end{lemma}
\begin{proof}
If $\gamma$ and $\gamma'$ are tangent, then there exists a point $(x,y) \in \gamma \cap \gamma'$ so the first two coordinates of $\beta(\gamma)$ and $\beta(\gamma')$ agree. 
Further, at a point of tangency we have that the slope of the tangent lines coincide for $\gamma$ and $\gamma'$, so the final coordinate is also equal. Hence $\beta(\gamma) \cap \beta(\gamma') \neq \emptyset$.

In the other direction, assume there exists some $(x,y,z) \in \beta(\gamma) \cap \beta(\gamma')$. Clearly $(x,y) \in \gamma \cap \gamma'$, and again the slope of the tangent lines at this point are equal, hence we conclude
$\gamma$ is tangent to $\gamma'$.
\end{proof}
This one-to-one correspondence between tangencies in $\RR^2$ and incidences in $\RR^3$ is the key idea behind the proof. To use the polynomial method here, we will need to ensure our tangencies
are sufficiently uniformly distributed among our circles. We can refine our collection such that this is the case by the following lemma.

\begin{lemma}
    Let $\CE$ be a collection of $N$ circles and suppose that  $\tau(\CE) \gtrsim N^{\alpha}$. 
    Then we can refine our set such that every circle in $\CE' \subset \CE$ is tangent to $\gtrsim N^{\alpha -1}$ circles.
\end{lemma}

\begin{proof}
Let $\tau(\CE)$ be the set of tangencies of the circles in $\CE$. Let $\CE_0 = \CE$. Take a circle $\gamma \in \CE_0$ such that $|\{(\gamma, \gamma') \ |\ (\gamma, \gamma') \in \tau(\CE_0)\}| < c_1 N^{1/2}$
and discard it. We label our new refined collection as $\CE_1$. From this collection, again take a circle $\gamma$ such that $|\{(\gamma, \gamma') \ |\ (\gamma, \gamma') \in \tau(\CE_1)\}| < c_1 N^{1/2}$ and discard it. 

After repeating this process $M$ times until there are no more circles that satisfy our criteria, 
at each step removing a circle that contributes only a small number of tangencies, we attain a collection $\CE_M$.
 We claim that $\tau(\CE_M) \gtrsim N^{\alpha}$, and that $\CE_M \neq \emptyset$.

For the first claim, observe that at each step $i$ we are reducing $\tau(\CE_i)$ by at most $c_1 N^{\alpha -1}$.  Thus,
\begin{align*}
    |\tau(\CE_M)| &\geq |\tau(\CE_0)| - M c_1 N^{\alpha -1} \\
    &> c_0 N^{\alpha} -  M c_1 N^{\alpha -1} \\ 
    &> c_0 N^{\alpha} - c_1 N^{\alpha} \\
    |\tau(\CE_M)| &> \frac{c_0}{2} N^{\alpha}. \qquad (\text{by choosing } c_1 = c_0/2 )
\end{align*}
We must now check that we have not removed every circle from our collection.\todo{disagree with correction here, tangencies is not necessarily non-zero, why else get this lower bound?} We have the trivial inequality $|\tau(\CE_M)| \leq |\CE_M|^{2}$. 
Combining this with the result above, we attain $|\CE_M| \gtrsim |N|^{\alpha/2}$.
\end{proof}
In a similar fashion to our arguments in the Kakeya problem we will be arguing by contradiction, however here we will be arguing that if the zero set of a polynomial contains too many of a certain first type of object, 
it must contain many more of a second kind of object. 

We will need a result from Bezout which bounds the number of intersections between curves in $\RR^3$.

\begin{lemma}[Bezout's Theorem for Plane Algebraic Curves]
   Let $Z_1$ and $Z_2$ be the zero sets of two polynomials over $\RR[X,Y]$. Then the following holds:
   \[
       |\{ x | x \in Z_1 \cap Z_2 \}| \leq \deg Z_1 \deg Z_2
   \]
    \label{lem:Bezout}
\end{lemma}


We can now prove the main theorem. 
\begin{proof}
    Given an arbitrary collection of circles $\CE$ with $\gtrsim N^{3/2}$ tangencies, we can reduce to a collection $\Gamma$ where each circle is tangent to at least $\sim N^{1/2}$ other circles using the previous lemma. 
    After applying a small rotation, we can assume that the tangent line at each point of tangency does not point vertically in the $y$-direction.
    Let $\beta (\Gamma) = \{ \beta(\gamma) : \gamma \in \Gamma \}$, where $\beta$ is the lifting transform defined earlier,
    Recall from Lemma \ref{lem:beta-lift} that two circles $\gamma_1$ and $\gamma_2$ are tangent if and only if $\beta(\gamma_1) \cap \beta(\gamma_2) \neq \emptyset$.

    Suppose $(x,y,z) \in \beta(\gamma_1) \cap \beta(\gamma_2) $ for some $\gamma_1 \neq \gamma_2$. 
    Then $$(0,0,1) \in \text{span} \left( T_{(x,y,z) }\beta (\gamma_1), T_{(x,y,z) }\beta (\gamma_2)\right).$$
    In other words, at the intersection of $\beta(\gamma_1)$ and $\beta(\gamma_2)$ their tangent vectors span a vertical subspace of $\RR^3$. 
    We can establish this by examining a parameterisation of $\gamma_1$ and $\gamma_2$ in the neighbourhood of $(x,y)$.
    Define $f_i (t), \ i \in \{1,2\}$ such that $(t+x, f_i(t))$ is a parameterisation of $\gamma_i$ in the neighbourhood of $(x,y)$ for all $t$ in a small neighbourhood of 0. In particular $f_i(0) = y$.
    Since $\gamma_1$ is tangent to $\gamma_2$ at $(x,y)$, $\frac{df_1}{dt}(0) = \frac{df_2}{dt}(0)$. 
    Since $\gamma_1$ and $\gamma_2$ are distinct quadratic curves, $\frac{d^2f_1}{dt^2}(0) \neq \frac{d^2f_2}{dt^2}(0)$. \todo{this was intuitive, not anymore :(}
    In the neighbourhood of $(x,y,z)$, $\beta(\gamma_i)$ is parametrised by $\left(t,f_i (t) ,\frac{df_1}{dt}(t) \right)$ as the slope of the tangent to the circle is given by $\frac{df_1}{dt}(t)$.\todo{I'm not convinced in this explan.} 
     It follows that the tangent vector
    $\left(1,\frac{df_i}{dt}(0), \frac{d^2f_i}{dt^2} (0) \right)$ generates the vertical space $T_{(x,y,z)} \beta(\gamma_i)$. Thus 
    \begin{align*} (0,0,1) &\in \text{span}\left( \left(1,\frac{df_1}{dt}(0), \frac{d^2f_1}{dt^2} (0) \right) - \left(1,\frac{df_2}{dt}(0), \frac{d^2f_2}{dt^2} (0) \right) \right)
    \\ &\subset \text{span} \left( T_{(x,y,z) }\beta (\gamma_1), T_{(x,y,z) }\beta (\gamma_2)\right). 
    \end{align*}

    We will now interpolate all points of intersection with a polynomial of suitably low degree, and show that if this contains too many points it will lead to a contradiction.
    Let $P \in \RR[x,y,z]$ be a non-zero polynomial of minimal degree that vanishes on all intersections between the curves in $\beta (\Gamma)$. The degree of $P$ is $\sim N^{1/2}$. 
    By our result above, if $(x,y,z)$ is a point where two curves from $\beta (\Gamma)$ intersect, then $\partial_z P (x,y,z)  =0 $. Thus since each $\gamma \in \Gamma$ is 
    tangent to $\gtrsim N^{1/2}$, and each of these tangencies occur at a distinct point, we have that $\partial_z P$ vanishes at $\gtrsim N^{1/2}$ points on each curve in $\beta (\Gamma)$.
    By \hyperref[lem:Bezout]{Bézout's theorem} we have that $\partial_z P$ vanishes on all curves in $\Gamma$ as:
    \[
    \deg (\partial_z P) \deg (\gamma) \sim  (N^{1/2}) \gtrsim \# \{\partial_z P \cap \gamma\} \sim (N^{1/2}) .    
    \] Since $P$ was the non-zero polynomial of minimal degree that vanishes on all the curves in $\beta (\Gamma)$, we must conclude 
    $\partial_z P$= 0. We have then that $P(x,y,z) = Q(x,y)$ for some $Q \in \RR[x,y]$ with degree $\sim N^{1/2}$. 
    But this implies that each of the $N$ circles in $\Gamma$ must be in $Z(Q)$. This is a contradiction, as $Q$ has degree $\sim N^{1/2}$ whereas $\cup \gamma$ has degree $2N$.
    We conclude that $\Gamma$ has fewer than $N^{3/2}$ tangencies.
\end{proof}







\bibliographystyle{unsrt}
\bibliography{fyp}
\addcontentsline{toc}{chapter}{References}

\clearpage
\appendix 

\phantomsection

\addcontentsline{toc}{chapter}{Appendices}
\chapter{Proof of Bézout's Lemma}
\chapter{Proof of Borsuk-Ulam Theorem}
    \begin{theorem}[Borsuk-Ulam]
        A map $\phi$ is said to be antipodal if it obeys $\phi (-x) = -\phi(x)$ for all $x$ in its domain. Suppose $\phi: \SP^N \to \RR^N$ is a continuous antipodal mapping. 
        Then the image of $\phi$ contains 0. \label{thm:Borsuk-Ulam}
    \end{theorem}
We present here a combinational proof due to Matousek.\cite{matouvsek2003using} 

Let $\|x\|_1$ be the $L_1$ norm of $x$. Define $B^n$ as the unit ball with respect to the $L_1$ norm, 
that is: $B^n = \{x\in \RR^n \ \big | \ \|x\|_1 \leq 1 \}$.

A simplex is the convex hull of an affinely independent set in $\RR^n$. A family of simplexes $\Delta = \{\sigma_1,\sigma_2, \dots, \sigma_m\}$ is called a \textbf{simplicial complex} if the following conditions hold:
\begin{enumerate}
    \item Each non-empty face of any simplex $\sigma \in \Delta$ is also a simplex of $\Delta$.
    \item $\sigma_1, \sigma_2 \in \Delta \implies \sigma_1 \cap \sigma_2$ is a face of both $\sigma_1$ and $\sigma_2$.
\end{enumerate}
A simplicial complex $T$ is a \textbf{special triangulation} of $B^n$ if:
\begin{enumerate}
    \item $\|T\| = B^n$
    \item $T$ is a refinement of the triangulation of $B^n$ given by cutting the coordinate hyperplanes. (In other words, no simplex of $T$ spans over a boundary of an orthant)
    \item $T$ is symmetrical around the origin.
\end{enumerate}

\begin{lemma}[Tucker Lemma]
    Let the vertices of an arbitrary special triangulation $T$ be denoted by labels $\text{lab} (u) \in \{\pm1, \pm2,\dots ,\pm n\}$
    in such a way that the vertices $u \in \partial B^n$ on the boundary satisfies $\text{lab} (-u) = -\text{lab} (u)$. Then there exists a 1-simplex (an edge) in $T$ which is complimentary, that is its two vertices $x,x'$ satisfy $\text{lab} (-x) = -\text{lab} (x')$.
\end{lemma}

\clearpage
\phantomsection
\addcontentsline{toc}{chapter}{Index}
\printindex

\end{document}
